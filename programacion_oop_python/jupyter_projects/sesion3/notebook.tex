
% Default to the notebook output style

    


% Inherit from the specified cell style.




    
\documentclass[11pt]{article}

    
    
    \usepackage[T1]{fontenc}
    % Nicer default font (+ math font) than Computer Modern for most use cases
    \usepackage{mathpazo}

    % Basic figure setup, for now with no caption control since it's done
    % automatically by Pandoc (which extracts ![](path) syntax from Markdown).
    \usepackage{graphicx}
    % We will generate all images so they have a width \maxwidth. This means
    % that they will get their normal width if they fit onto the page, but
    % are scaled down if they would overflow the margins.
    \makeatletter
    \def\maxwidth{\ifdim\Gin@nat@width>\linewidth\linewidth
    \else\Gin@nat@width\fi}
    \makeatother
    \let\Oldincludegraphics\includegraphics
    % Set max figure width to be 80% of text width, for now hardcoded.
    \renewcommand{\includegraphics}[1]{\Oldincludegraphics[width=.8\maxwidth]{#1}}
    % Ensure that by default, figures have no caption (until we provide a
    % proper Figure object with a Caption API and a way to capture that
    % in the conversion process - todo).
    \usepackage{caption}
    \DeclareCaptionLabelFormat{nolabel}{}
    \captionsetup{labelformat=nolabel}

    \usepackage{adjustbox} % Used to constrain images to a maximum size 
    \usepackage{xcolor} % Allow colors to be defined
    \usepackage{enumerate} % Needed for markdown enumerations to work
    \usepackage{geometry} % Used to adjust the document margins
    \usepackage{amsmath} % Equations
    \usepackage{amssymb} % Equations
    \usepackage{textcomp} % defines textquotesingle
    % Hack from http://tex.stackexchange.com/a/47451/13684:
    \AtBeginDocument{%
        \def\PYZsq{\textquotesingle}% Upright quotes in Pygmentized code
    }
    \usepackage{upquote} % Upright quotes for verbatim code
    \usepackage{eurosym} % defines \euro
    \usepackage[mathletters]{ucs} % Extended unicode (utf-8) support
    \usepackage[utf8x]{inputenc} % Allow utf-8 characters in the tex document
    \usepackage{fancyvrb} % verbatim replacement that allows latex
    \usepackage{grffile} % extends the file name processing of package graphics 
                         % to support a larger range 
    % The hyperref package gives us a pdf with properly built
    % internal navigation ('pdf bookmarks' for the table of contents,
    % internal cross-reference links, web links for URLs, etc.)
    \usepackage{hyperref}
    \usepackage{longtable} % longtable support required by pandoc >1.10
    \usepackage{booktabs}  % table support for pandoc > 1.12.2
    \usepackage[inline]{enumitem} % IRkernel/repr support (it uses the enumerate* environment)
    \usepackage[normalem]{ulem} % ulem is needed to support strikethroughs (\sout)
                                % normalem makes italics be italics, not underlines
    

    
    
    % Colors for the hyperref package
    \definecolor{urlcolor}{rgb}{0,.145,.698}
    \definecolor{linkcolor}{rgb}{.71,0.21,0.01}
    \definecolor{citecolor}{rgb}{.12,.54,.11}

    % ANSI colors
    \definecolor{ansi-black}{HTML}{3E424D}
    \definecolor{ansi-black-intense}{HTML}{282C36}
    \definecolor{ansi-red}{HTML}{E75C58}
    \definecolor{ansi-red-intense}{HTML}{B22B31}
    \definecolor{ansi-green}{HTML}{00A250}
    \definecolor{ansi-green-intense}{HTML}{007427}
    \definecolor{ansi-yellow}{HTML}{DDB62B}
    \definecolor{ansi-yellow-intense}{HTML}{B27D12}
    \definecolor{ansi-blue}{HTML}{208FFB}
    \definecolor{ansi-blue-intense}{HTML}{0065CA}
    \definecolor{ansi-magenta}{HTML}{D160C4}
    \definecolor{ansi-magenta-intense}{HTML}{A03196}
    \definecolor{ansi-cyan}{HTML}{60C6C8}
    \definecolor{ansi-cyan-intense}{HTML}{258F8F}
    \definecolor{ansi-white}{HTML}{C5C1B4}
    \definecolor{ansi-white-intense}{HTML}{A1A6B2}

    % commands and environments needed by pandoc snippets
    % extracted from the output of `pandoc -s`
    \providecommand{\tightlist}{%
      \setlength{\itemsep}{0pt}\setlength{\parskip}{0pt}}
    \DefineVerbatimEnvironment{Highlighting}{Verbatim}{commandchars=\\\{\}}
    % Add ',fontsize=\small' for more characters per line
    \newenvironment{Shaded}{}{}
    \newcommand{\KeywordTok}[1]{\textcolor[rgb]{0.00,0.44,0.13}{\textbf{{#1}}}}
    \newcommand{\DataTypeTok}[1]{\textcolor[rgb]{0.56,0.13,0.00}{{#1}}}
    \newcommand{\DecValTok}[1]{\textcolor[rgb]{0.25,0.63,0.44}{{#1}}}
    \newcommand{\BaseNTok}[1]{\textcolor[rgb]{0.25,0.63,0.44}{{#1}}}
    \newcommand{\FloatTok}[1]{\textcolor[rgb]{0.25,0.63,0.44}{{#1}}}
    \newcommand{\CharTok}[1]{\textcolor[rgb]{0.25,0.44,0.63}{{#1}}}
    \newcommand{\StringTok}[1]{\textcolor[rgb]{0.25,0.44,0.63}{{#1}}}
    \newcommand{\CommentTok}[1]{\textcolor[rgb]{0.38,0.63,0.69}{\textit{{#1}}}}
    \newcommand{\OtherTok}[1]{\textcolor[rgb]{0.00,0.44,0.13}{{#1}}}
    \newcommand{\AlertTok}[1]{\textcolor[rgb]{1.00,0.00,0.00}{\textbf{{#1}}}}
    \newcommand{\FunctionTok}[1]{\textcolor[rgb]{0.02,0.16,0.49}{{#1}}}
    \newcommand{\RegionMarkerTok}[1]{{#1}}
    \newcommand{\ErrorTok}[1]{\textcolor[rgb]{1.00,0.00,0.00}{\textbf{{#1}}}}
    \newcommand{\NormalTok}[1]{{#1}}
    
    % Additional commands for more recent versions of Pandoc
    \newcommand{\ConstantTok}[1]{\textcolor[rgb]{0.53,0.00,0.00}{{#1}}}
    \newcommand{\SpecialCharTok}[1]{\textcolor[rgb]{0.25,0.44,0.63}{{#1}}}
    \newcommand{\VerbatimStringTok}[1]{\textcolor[rgb]{0.25,0.44,0.63}{{#1}}}
    \newcommand{\SpecialStringTok}[1]{\textcolor[rgb]{0.73,0.40,0.53}{{#1}}}
    \newcommand{\ImportTok}[1]{{#1}}
    \newcommand{\DocumentationTok}[1]{\textcolor[rgb]{0.73,0.13,0.13}{\textit{{#1}}}}
    \newcommand{\AnnotationTok}[1]{\textcolor[rgb]{0.38,0.63,0.69}{\textbf{\textit{{#1}}}}}
    \newcommand{\CommentVarTok}[1]{\textcolor[rgb]{0.38,0.63,0.69}{\textbf{\textit{{#1}}}}}
    \newcommand{\VariableTok}[1]{\textcolor[rgb]{0.10,0.09,0.49}{{#1}}}
    \newcommand{\ControlFlowTok}[1]{\textcolor[rgb]{0.00,0.44,0.13}{\textbf{{#1}}}}
    \newcommand{\OperatorTok}[1]{\textcolor[rgb]{0.40,0.40,0.40}{{#1}}}
    \newcommand{\BuiltInTok}[1]{{#1}}
    \newcommand{\ExtensionTok}[1]{{#1}}
    \newcommand{\PreprocessorTok}[1]{\textcolor[rgb]{0.74,0.48,0.00}{{#1}}}
    \newcommand{\AttributeTok}[1]{\textcolor[rgb]{0.49,0.56,0.16}{{#1}}}
    \newcommand{\InformationTok}[1]{\textcolor[rgb]{0.38,0.63,0.69}{\textbf{\textit{{#1}}}}}
    \newcommand{\WarningTok}[1]{\textcolor[rgb]{0.38,0.63,0.69}{\textbf{\textit{{#1}}}}}
    
    
    % Define a nice break command that doesn't care if a line doesn't already
    % exist.
    \def\br{\hspace*{\fill} \\* }
    % Math Jax compatability definitions
    \def\gt{>}
    \def\lt{<}
    % Document parameters
    \title{POO con Python IV}
    
    
    

    % Pygments definitions
    
\makeatletter
\def\PY@reset{\let\PY@it=\relax \let\PY@bf=\relax%
    \let\PY@ul=\relax \let\PY@tc=\relax%
    \let\PY@bc=\relax \let\PY@ff=\relax}
\def\PY@tok#1{\csname PY@tok@#1\endcsname}
\def\PY@toks#1+{\ifx\relax#1\empty\else%
    \PY@tok{#1}\expandafter\PY@toks\fi}
\def\PY@do#1{\PY@bc{\PY@tc{\PY@ul{%
    \PY@it{\PY@bf{\PY@ff{#1}}}}}}}
\def\PY#1#2{\PY@reset\PY@toks#1+\relax+\PY@do{#2}}

\expandafter\def\csname PY@tok@w\endcsname{\def\PY@tc##1{\textcolor[rgb]{0.73,0.73,0.73}{##1}}}
\expandafter\def\csname PY@tok@c\endcsname{\let\PY@it=\textit\def\PY@tc##1{\textcolor[rgb]{0.25,0.50,0.50}{##1}}}
\expandafter\def\csname PY@tok@cp\endcsname{\def\PY@tc##1{\textcolor[rgb]{0.74,0.48,0.00}{##1}}}
\expandafter\def\csname PY@tok@k\endcsname{\let\PY@bf=\textbf\def\PY@tc##1{\textcolor[rgb]{0.00,0.50,0.00}{##1}}}
\expandafter\def\csname PY@tok@kp\endcsname{\def\PY@tc##1{\textcolor[rgb]{0.00,0.50,0.00}{##1}}}
\expandafter\def\csname PY@tok@kt\endcsname{\def\PY@tc##1{\textcolor[rgb]{0.69,0.00,0.25}{##1}}}
\expandafter\def\csname PY@tok@o\endcsname{\def\PY@tc##1{\textcolor[rgb]{0.40,0.40,0.40}{##1}}}
\expandafter\def\csname PY@tok@ow\endcsname{\let\PY@bf=\textbf\def\PY@tc##1{\textcolor[rgb]{0.67,0.13,1.00}{##1}}}
\expandafter\def\csname PY@tok@nb\endcsname{\def\PY@tc##1{\textcolor[rgb]{0.00,0.50,0.00}{##1}}}
\expandafter\def\csname PY@tok@nf\endcsname{\def\PY@tc##1{\textcolor[rgb]{0.00,0.00,1.00}{##1}}}
\expandafter\def\csname PY@tok@nc\endcsname{\let\PY@bf=\textbf\def\PY@tc##1{\textcolor[rgb]{0.00,0.00,1.00}{##1}}}
\expandafter\def\csname PY@tok@nn\endcsname{\let\PY@bf=\textbf\def\PY@tc##1{\textcolor[rgb]{0.00,0.00,1.00}{##1}}}
\expandafter\def\csname PY@tok@ne\endcsname{\let\PY@bf=\textbf\def\PY@tc##1{\textcolor[rgb]{0.82,0.25,0.23}{##1}}}
\expandafter\def\csname PY@tok@nv\endcsname{\def\PY@tc##1{\textcolor[rgb]{0.10,0.09,0.49}{##1}}}
\expandafter\def\csname PY@tok@no\endcsname{\def\PY@tc##1{\textcolor[rgb]{0.53,0.00,0.00}{##1}}}
\expandafter\def\csname PY@tok@nl\endcsname{\def\PY@tc##1{\textcolor[rgb]{0.63,0.63,0.00}{##1}}}
\expandafter\def\csname PY@tok@ni\endcsname{\let\PY@bf=\textbf\def\PY@tc##1{\textcolor[rgb]{0.60,0.60,0.60}{##1}}}
\expandafter\def\csname PY@tok@na\endcsname{\def\PY@tc##1{\textcolor[rgb]{0.49,0.56,0.16}{##1}}}
\expandafter\def\csname PY@tok@nt\endcsname{\let\PY@bf=\textbf\def\PY@tc##1{\textcolor[rgb]{0.00,0.50,0.00}{##1}}}
\expandafter\def\csname PY@tok@nd\endcsname{\def\PY@tc##1{\textcolor[rgb]{0.67,0.13,1.00}{##1}}}
\expandafter\def\csname PY@tok@s\endcsname{\def\PY@tc##1{\textcolor[rgb]{0.73,0.13,0.13}{##1}}}
\expandafter\def\csname PY@tok@sd\endcsname{\let\PY@it=\textit\def\PY@tc##1{\textcolor[rgb]{0.73,0.13,0.13}{##1}}}
\expandafter\def\csname PY@tok@si\endcsname{\let\PY@bf=\textbf\def\PY@tc##1{\textcolor[rgb]{0.73,0.40,0.53}{##1}}}
\expandafter\def\csname PY@tok@se\endcsname{\let\PY@bf=\textbf\def\PY@tc##1{\textcolor[rgb]{0.73,0.40,0.13}{##1}}}
\expandafter\def\csname PY@tok@sr\endcsname{\def\PY@tc##1{\textcolor[rgb]{0.73,0.40,0.53}{##1}}}
\expandafter\def\csname PY@tok@ss\endcsname{\def\PY@tc##1{\textcolor[rgb]{0.10,0.09,0.49}{##1}}}
\expandafter\def\csname PY@tok@sx\endcsname{\def\PY@tc##1{\textcolor[rgb]{0.00,0.50,0.00}{##1}}}
\expandafter\def\csname PY@tok@m\endcsname{\def\PY@tc##1{\textcolor[rgb]{0.40,0.40,0.40}{##1}}}
\expandafter\def\csname PY@tok@gh\endcsname{\let\PY@bf=\textbf\def\PY@tc##1{\textcolor[rgb]{0.00,0.00,0.50}{##1}}}
\expandafter\def\csname PY@tok@gu\endcsname{\let\PY@bf=\textbf\def\PY@tc##1{\textcolor[rgb]{0.50,0.00,0.50}{##1}}}
\expandafter\def\csname PY@tok@gd\endcsname{\def\PY@tc##1{\textcolor[rgb]{0.63,0.00,0.00}{##1}}}
\expandafter\def\csname PY@tok@gi\endcsname{\def\PY@tc##1{\textcolor[rgb]{0.00,0.63,0.00}{##1}}}
\expandafter\def\csname PY@tok@gr\endcsname{\def\PY@tc##1{\textcolor[rgb]{1.00,0.00,0.00}{##1}}}
\expandafter\def\csname PY@tok@ge\endcsname{\let\PY@it=\textit}
\expandafter\def\csname PY@tok@gs\endcsname{\let\PY@bf=\textbf}
\expandafter\def\csname PY@tok@gp\endcsname{\let\PY@bf=\textbf\def\PY@tc##1{\textcolor[rgb]{0.00,0.00,0.50}{##1}}}
\expandafter\def\csname PY@tok@go\endcsname{\def\PY@tc##1{\textcolor[rgb]{0.53,0.53,0.53}{##1}}}
\expandafter\def\csname PY@tok@gt\endcsname{\def\PY@tc##1{\textcolor[rgb]{0.00,0.27,0.87}{##1}}}
\expandafter\def\csname PY@tok@err\endcsname{\def\PY@bc##1{\setlength{\fboxsep}{0pt}\fcolorbox[rgb]{1.00,0.00,0.00}{1,1,1}{\strut ##1}}}
\expandafter\def\csname PY@tok@kc\endcsname{\let\PY@bf=\textbf\def\PY@tc##1{\textcolor[rgb]{0.00,0.50,0.00}{##1}}}
\expandafter\def\csname PY@tok@kd\endcsname{\let\PY@bf=\textbf\def\PY@tc##1{\textcolor[rgb]{0.00,0.50,0.00}{##1}}}
\expandafter\def\csname PY@tok@kn\endcsname{\let\PY@bf=\textbf\def\PY@tc##1{\textcolor[rgb]{0.00,0.50,0.00}{##1}}}
\expandafter\def\csname PY@tok@kr\endcsname{\let\PY@bf=\textbf\def\PY@tc##1{\textcolor[rgb]{0.00,0.50,0.00}{##1}}}
\expandafter\def\csname PY@tok@bp\endcsname{\def\PY@tc##1{\textcolor[rgb]{0.00,0.50,0.00}{##1}}}
\expandafter\def\csname PY@tok@fm\endcsname{\def\PY@tc##1{\textcolor[rgb]{0.00,0.00,1.00}{##1}}}
\expandafter\def\csname PY@tok@vc\endcsname{\def\PY@tc##1{\textcolor[rgb]{0.10,0.09,0.49}{##1}}}
\expandafter\def\csname PY@tok@vg\endcsname{\def\PY@tc##1{\textcolor[rgb]{0.10,0.09,0.49}{##1}}}
\expandafter\def\csname PY@tok@vi\endcsname{\def\PY@tc##1{\textcolor[rgb]{0.10,0.09,0.49}{##1}}}
\expandafter\def\csname PY@tok@vm\endcsname{\def\PY@tc##1{\textcolor[rgb]{0.10,0.09,0.49}{##1}}}
\expandafter\def\csname PY@tok@sa\endcsname{\def\PY@tc##1{\textcolor[rgb]{0.73,0.13,0.13}{##1}}}
\expandafter\def\csname PY@tok@sb\endcsname{\def\PY@tc##1{\textcolor[rgb]{0.73,0.13,0.13}{##1}}}
\expandafter\def\csname PY@tok@sc\endcsname{\def\PY@tc##1{\textcolor[rgb]{0.73,0.13,0.13}{##1}}}
\expandafter\def\csname PY@tok@dl\endcsname{\def\PY@tc##1{\textcolor[rgb]{0.73,0.13,0.13}{##1}}}
\expandafter\def\csname PY@tok@s2\endcsname{\def\PY@tc##1{\textcolor[rgb]{0.73,0.13,0.13}{##1}}}
\expandafter\def\csname PY@tok@sh\endcsname{\def\PY@tc##1{\textcolor[rgb]{0.73,0.13,0.13}{##1}}}
\expandafter\def\csname PY@tok@s1\endcsname{\def\PY@tc##1{\textcolor[rgb]{0.73,0.13,0.13}{##1}}}
\expandafter\def\csname PY@tok@mb\endcsname{\def\PY@tc##1{\textcolor[rgb]{0.40,0.40,0.40}{##1}}}
\expandafter\def\csname PY@tok@mf\endcsname{\def\PY@tc##1{\textcolor[rgb]{0.40,0.40,0.40}{##1}}}
\expandafter\def\csname PY@tok@mh\endcsname{\def\PY@tc##1{\textcolor[rgb]{0.40,0.40,0.40}{##1}}}
\expandafter\def\csname PY@tok@mi\endcsname{\def\PY@tc##1{\textcolor[rgb]{0.40,0.40,0.40}{##1}}}
\expandafter\def\csname PY@tok@il\endcsname{\def\PY@tc##1{\textcolor[rgb]{0.40,0.40,0.40}{##1}}}
\expandafter\def\csname PY@tok@mo\endcsname{\def\PY@tc##1{\textcolor[rgb]{0.40,0.40,0.40}{##1}}}
\expandafter\def\csname PY@tok@ch\endcsname{\let\PY@it=\textit\def\PY@tc##1{\textcolor[rgb]{0.25,0.50,0.50}{##1}}}
\expandafter\def\csname PY@tok@cm\endcsname{\let\PY@it=\textit\def\PY@tc##1{\textcolor[rgb]{0.25,0.50,0.50}{##1}}}
\expandafter\def\csname PY@tok@cpf\endcsname{\let\PY@it=\textit\def\PY@tc##1{\textcolor[rgb]{0.25,0.50,0.50}{##1}}}
\expandafter\def\csname PY@tok@c1\endcsname{\let\PY@it=\textit\def\PY@tc##1{\textcolor[rgb]{0.25,0.50,0.50}{##1}}}
\expandafter\def\csname PY@tok@cs\endcsname{\let\PY@it=\textit\def\PY@tc##1{\textcolor[rgb]{0.25,0.50,0.50}{##1}}}

\def\PYZbs{\char`\\}
\def\PYZus{\char`\_}
\def\PYZob{\char`\{}
\def\PYZcb{\char`\}}
\def\PYZca{\char`\^}
\def\PYZam{\char`\&}
\def\PYZlt{\char`\<}
\def\PYZgt{\char`\>}
\def\PYZsh{\char`\#}
\def\PYZpc{\char`\%}
\def\PYZdl{\char`\$}
\def\PYZhy{\char`\-}
\def\PYZsq{\char`\'}
\def\PYZdq{\char`\"}
\def\PYZti{\char`\~}
% for compatibility with earlier versions
\def\PYZat{@}
\def\PYZlb{[}
\def\PYZrb{]}
\makeatother


    % Exact colors from NB
    \definecolor{incolor}{rgb}{0.0, 0.0, 0.5}
    \definecolor{outcolor}{rgb}{0.545, 0.0, 0.0}



    
    % Prevent overflowing lines due to hard-to-break entities
    \sloppy 
    % Setup hyperref package
    \hypersetup{
      breaklinks=true,  % so long urls are correctly broken across lines
      colorlinks=true,
      urlcolor=urlcolor,
      linkcolor=linkcolor,
      citecolor=citecolor,
      }
    % Slightly bigger margins than the latex defaults
    
    \geometry{verbose,tmargin=1in,bmargin=1in,lmargin=1in,rmargin=1in}
    
    

    \begin{document}
    
    
    \maketitle
    
    

    
    \section{}\label{section}

POO con Python

Módulo 2: Numpy II

Yuri Coicca, M.Sc. Facultad de Ciencias Pre-Maestria en Ciencia de la
Computación

    \begin{center}\rule{0.5\linewidth}{\linethickness}\end{center}

Índice

\begin{itemize}
\tightlist
\item
  Section \ref{section6}
\item
  Section \ref{section7}

  \begin{itemize}
  \tightlist
  \item
    Section \ref{section71}
  \item
    Section \ref{section72}\\
  \item
    Section \ref{section73}\\
  \item
    Section \ref{section74}
  \item
    Section \ref{section75}
  \item
    Section \ref{section76}
  \end{itemize}
\item
  Section \ref{section8}

  \begin{itemize}
  \tightlist
  \item
    Section \ref{section81}
  \item
    Section \ref{section82}
  \item
    Section \ref{section83}\\
  \item
    Section \ref{section84}\\
  \end{itemize}
\item
  Section \ref{section9}

  \begin{itemize}
  \tightlist
  \item
    Section \ref{section91}\\
  \item
    Section \ref{section92}\\
  \item
    Section \ref{section93}\\
  \item
    Section \ref{section94}\\
  \item
    Section \ref{section95}\\
  \item
    Section \ref{section96}\\
  \item
    Section \ref{section97}
  \item
    Section \ref{section98}
  \end{itemize}
\end{itemize}

    \begin{center}\rule{0.5\linewidth}{\linethickness}\end{center}

 6. Vectorización

Una de las principales ventajas que aporta \emph{Numpy} es que muchas
funciones se implementan internamente de forma \emph{vectorizada}, lo
que supone un aumento de la eficiencia \emph{muy importante} (varios
órdenes de magnitud) con respecto a las operaciones secuenciales.

    \begin{Verbatim}[commandchars=\\\{\}]
{\color{incolor}In [{\color{incolor} }]:} \PY{k+kn}{import} \PY{n+nn}{numpy} \PY{k}{as} \PY{n+nn}{np}
        
        \PY{c+c1}{\PYZsh{} Crea un vector de 100000 números aleatorios.}
        \PY{n}{gran\PYZus{}vector} \PY{o}{=} \PY{n}{np}\PY{o}{.}\PY{n}{random}\PY{o}{.}\PY{n}{randint}\PY{p}{(}\PY{l+m+mi}{0}\PY{p}{,}\PY{l+m+mi}{1000}\PY{p}{,}\PY{l+m+mi}{1000000}\PY{p}{)}
        \PY{n+nb}{print}\PY{p}{(}\PY{n}{gran\PYZus{}vector}\PY{o}{.}\PY{n}{shape}\PY{p}{)}
\end{Verbatim}


    Mediante la palabra clave o \emph{magic} \texttt{timeit}, se indica a
\emph{Jupyter} que ejecute un número determinado de veces (en este caso
10) el código que aparece en la celda y mida el tiempo medio que tarda y
la desviación típoca. Por defecto, repite esta operación 7 veces, y
devuelve el mejor tiempo medio.

En esta primera llamada, se obtiene la suma de los elementos del vector
mediante un bucle.

    \begin{Verbatim}[commandchars=\\\{\}]
{\color{incolor}In [{\color{incolor} }]:} \PY{o}{\PYZpc{}\PYZpc{}}\PY{k}{timeit} \PYZhy{}n 10
        suma = 0
        for elem in gran\PYZus{}vector:
            suma+=elem
\end{Verbatim}


    En esta llamada se suman los elementos mediante la función vectorizada
\texttt{np.sum()} (se verán las funciones en el siguiente apartado).
Puede verse que el tiempo necesario se reduce en varios órdenes de
magnitud.

    \begin{Verbatim}[commandchars=\\\{\}]
{\color{incolor}In [{\color{incolor} }]:} \PY{o}{\PYZpc{}\PYZpc{}}\PY{k}{timeit} \PYZhy{}n 10
        suma = np.sum(gran\PYZus{}vector)
\end{Verbatim}


     

    \begin{center}\rule{0.5\linewidth}{\linethickness}\end{center}

 7. Operadores y funciones \emph{universales}

    \emph{NumPy} implementa numerosas funciones. Algunas de ellas,
denominadas \emph{universales} (\emph{ufunc}), se aplican de manera
eficiente (son implementaciones \emph{vectorizadas}) elemento por
elemento, y soportan algunas características como \emph{broadcasting}
(que es un mecanismo que permite más flexibilidad, y se verá más
adelante). También se pueden aplicar entre cada elemento de un array y
un escalar.

A continuación se describen algunas de las funciones universales de uso
más frecuente. Una lista completa de este tipo de funciones puede
consultarse
\href{https://docs.scipy.org/doc/numpy/reference/ufuncs.html\#available-ufuncs}{aquí}.
El resto de funciones disponibles (también las no universales) pueden
consultarse en la
\href{https://docs.scipy.org/doc/numpy/reference/index.html}{referencia
de Numpy}.

    \begin{center}\rule{0.5\linewidth}{\linethickness}\end{center}

 Funciones y operaciones aritméticas básicas

    \begin{Verbatim}[commandchars=\\\{\}]
{\color{incolor}In [{\color{incolor} }]:} \PY{n}{x} \PY{o}{=} \PY{n}{np}\PY{o}{.}\PY{n}{array}\PY{p}{(}\PY{p}{[}\PY{p}{[}\PY{l+m+mi}{1}\PY{p}{,}\PY{l+m+mi}{2}\PY{p}{,}\PY{l+m+mi}{3}\PY{p}{]}\PY{p}{,}\PY{p}{[}\PY{l+m+mi}{4}\PY{p}{,}\PY{l+m+mi}{5}\PY{p}{,}\PY{l+m+mi}{6}\PY{p}{]}\PY{p}{]}\PY{p}{,} \PY{n}{dtype}\PY{o}{=}\PY{n}{np}\PY{o}{.}\PY{n}{float64}\PY{p}{)}
        \PY{n}{y} \PY{o}{=} \PY{n}{np}\PY{o}{.}\PY{n}{array}\PY{p}{(}\PY{p}{[}\PY{p}{[}\PY{l+m+mi}{10}\PY{p}{,}\PY{l+m+mi}{10}\PY{p}{,}\PY{l+m+mi}{10}\PY{p}{]}\PY{p}{,}\PY{p}{[}\PY{l+m+mi}{20}\PY{p}{,}\PY{l+m+mi}{20}\PY{p}{,}\PY{l+m+mi}{20}\PY{p}{]}\PY{p}{]}\PY{p}{,} \PY{n}{dtype}\PY{o}{=}\PY{n}{np}\PY{o}{.}\PY{n}{float64}\PY{p}{)}
        
        \PY{n+nb}{print}\PY{p}{(}\PY{n}{x}\PY{p}{)}
        \PY{n+nb}{print}\PY{p}{(}\PY{n}{y}\PY{p}{)}
        \PY{n+nb}{print}\PY{p}{(}\PY{l+s+s2}{\PYZdq{}}\PY{l+s+s2}{\PYZhy{}\PYZhy{}\PYZhy{}}\PY{l+s+s2}{\PYZdq{}}\PY{p}{)}
        
        \PY{n+nb}{print}\PY{p}{(}\PY{n}{np}\PY{o}{.}\PY{n}{add}\PY{p}{(}\PY{n}{x}\PY{p}{,} \PY{n}{y}\PY{p}{)}\PY{p}{)}                           \PY{c+c1}{\PYZsh{} Suma}
        \PY{n+nb}{print}\PY{p}{(}\PY{l+s+s2}{\PYZdq{}}\PY{l+s+s2}{\PYZhy{}\PYZhy{}\PYZhy{}}\PY{l+s+s2}{\PYZdq{}}\PY{p}{)}
        \PY{n+nb}{print}\PY{p}{(}\PY{n}{np}\PY{o}{.}\PY{n}{subtract}\PY{p}{(}\PY{n}{y}\PY{p}{,}\PY{l+m+mi}{5}\PY{p}{)}\PY{p}{)}                       \PY{c+c1}{\PYZsh{} Resta (con escalar)}
        \PY{n+nb}{print}\PY{p}{(}\PY{l+s+s2}{\PYZdq{}}\PY{l+s+s2}{\PYZhy{}\PYZhy{}\PYZhy{}}\PY{l+s+s2}{\PYZdq{}}\PY{p}{)}
        \PY{n+nb}{print}\PY{p}{(}\PY{n}{np}\PY{o}{.}\PY{n}{multiply}\PY{p}{(}\PY{n}{x}\PY{p}{,}\PY{n}{y}\PY{p}{)}\PY{p}{)}                       \PY{c+c1}{\PYZsh{} Multiplicación}
        \PY{n+nb}{print}\PY{p}{(}\PY{l+s+s2}{\PYZdq{}}\PY{l+s+s2}{\PYZhy{}\PYZhy{}\PYZhy{}}\PY{l+s+s2}{\PYZdq{}}\PY{p}{)}
        \PY{n+nb}{print}\PY{p}{(}\PY{n}{np}\PY{o}{.}\PY{n}{divide}\PY{p}{(}\PY{n}{y}\PY{p}{,}\PY{n}{x}\PY{p}{)}\PY{p}{)}                         \PY{c+c1}{\PYZsh{} División}
        \PY{n+nb}{print}\PY{p}{(}\PY{l+s+s2}{\PYZdq{}}\PY{l+s+s2}{\PYZhy{}\PYZhy{}\PYZhy{}}\PY{l+s+s2}{\PYZdq{}}\PY{p}{)}
        \PY{n+nb}{print}\PY{p}{(}\PY{n}{np}\PY{o}{.}\PY{n}{mod}\PY{p}{(}\PY{n}{x}\PY{p}{,}\PY{l+m+mi}{2}\PY{p}{)}\PY{p}{)}                            \PY{c+c1}{\PYZsh{} Resto de la división entera}
        \PY{n+nb}{print}\PY{p}{(}\PY{l+s+s2}{\PYZdq{}}\PY{l+s+s2}{\PYZhy{}\PYZhy{}\PYZhy{}}\PY{l+s+s2}{\PYZdq{}}\PY{p}{)}
        \PY{n+nb}{print}\PY{p}{(}\PY{n}{np}\PY{o}{.}\PY{n}{power}\PY{p}{(}\PY{n}{y}\PY{p}{,}\PY{l+m+mi}{2}\PY{p}{)}\PY{p}{)}                          \PY{c+c1}{\PYZsh{} Potencia}
\end{Verbatim}


    Estas operaciones también están implementadas en forma de operador.

    \begin{Verbatim}[commandchars=\\\{\}]
{\color{incolor}In [{\color{incolor} }]:} \PY{n+nb}{print}\PY{p}{(}\PY{n}{x} \PY{o}{+} \PY{n}{y}\PY{p}{)}                     
        \PY{n+nb}{print}\PY{p}{(}\PY{n}{y} \PY{o}{\PYZhy{}} \PY{l+m+mi}{5}\PY{p}{)}                      
        \PY{n+nb}{print}\PY{p}{(}\PY{n}{x} \PY{o}{*} \PY{n}{y}\PY{p}{)}                    
        \PY{n+nb}{print}\PY{p}{(}\PY{n}{y} \PY{o}{/} \PY{n}{x}\PY{p}{)} 
        \PY{n+nb}{print}\PY{p}{(}\PY{n}{x} \PY{o}{\PYZpc{}} \PY{l+m+mi}{2}\PY{p}{)}
        \PY{n+nb}{print}\PY{p}{(}\PY{n}{y} \PY{o}{*}\PY{o}{*} \PY{l+m+mi}{2}\PY{p}{)}
\end{Verbatim}


    \begin{center}\rule{0.5\linewidth}{\linethickness}\end{center}

 Funciones matemáticas

\begin{itemize}
\tightlist
\item
  Trigonométricas: \texttt{sin()}, \texttt{cos()}, \texttt{tan()},
  \texttt{arcsin()}, \texttt{arccos()}, \texttt{arctan()},
  \texttt{degrees()}
\item
  Logaritmos y explonenciales: \texttt{log()}, \texttt{log10()},
  \texttt{log2()},\texttt{exp()}, \texttt{exp2()}
\item
  Otras: \texttt{square()}, \texttt{reciprocal()}
\item
  etc.
\end{itemize}

    \begin{Verbatim}[commandchars=\\\{\}]
{\color{incolor}In [{\color{incolor} }]:} \PY{k+kn}{import} \PY{n+nn}{numpy} \PY{k}{as} \PY{n+nn}{np}
\end{Verbatim}


    \begin{Verbatim}[commandchars=\\\{\}]
{\color{incolor}In [{\color{incolor} }]:} \PY{n}{np}\PY{o}{.}\PY{n}{linspace}\PY{p}{(}\PY{l+m+mi}{1}\PY{p}{,} \PY{l+m+mi}{10}\PY{p}{,} \PY{l+m+mi}{10}\PY{p}{)} \PY{o}{*} \PY{n}{np}\PY{o}{.}\PY{n}{linspace}\PY{p}{(}\PY{l+m+mi}{1}\PY{p}{,} \PY{l+m+mi}{10}\PY{p}{,} \PY{l+m+mi}{1}\PY{p}{)}
\end{Verbatim}


    \begin{Verbatim}[commandchars=\\\{\}]
{\color{incolor}In [{\color{incolor} }]:} \PY{n}{a} \PY{o}{=} \PY{n}{np}\PY{o}{.}\PY{n}{array}\PY{p}{(}\PY{p}{[}\PY{l+m+mi}{0}\PY{p}{,}\PY{l+m+mi}{30}\PY{p}{,}\PY{l+m+mi}{45}\PY{p}{,}\PY{l+m+mi}{60}\PY{p}{,}\PY{l+m+mi}{90}\PY{p}{]}\PY{p}{)} 
        \PY{n+nb}{print}\PY{p}{(}\PY{l+s+s2}{\PYZdq{}}\PY{l+s+s2}{Ángulos}\PY{l+s+s2}{\PYZdq{}}\PY{p}{)}
        \PY{n+nb}{print}\PY{p}{(}\PY{n}{a}\PY{p}{)}
        \PY{n}{arad} \PY{o}{=} \PY{n}{a}\PY{o}{*}\PY{n}{np}\PY{o}{.}\PY{n}{pi}\PY{o}{/}\PY{l+m+mi}{180}                 \PY{c+c1}{\PYZsh{} Convierte a radianes todos los elementos}
        \PY{n+nb}{print}\PY{p}{(}\PY{l+s+s2}{\PYZdq{}}\PY{l+s+s2}{En radianes}\PY{l+s+s2}{\PYZdq{}}\PY{p}{)}
        \PY{n+nb}{print}\PY{p}{(}\PY{n}{arad}\PY{p}{)}
        \PY{n+nb}{print}\PY{p}{(}\PY{l+s+s1}{\PYZsq{}}\PY{l+s+s1}{Seno:}\PY{l+s+s1}{\PYZsq{}}\PY{p}{)} 
        \PY{n}{s} \PY{o}{=} \PY{n}{np}\PY{o}{.}\PY{n}{sin}\PY{p}{(}\PY{n}{arad}\PY{p}{)}
        \PY{n+nb}{print}\PY{p}{(}\PY{n}{s}\PY{p}{)}                           \PY{c+c1}{\PYZsh{} Imprime el seno de cada elemento}
        
        \PY{n}{b} \PY{o}{=} \PY{n}{np}\PY{o}{.}\PY{n}{array}\PY{p}{(}\PY{p}{[}\PY{l+m+mi}{1}\PY{p}{,} \PY{l+m+mi}{10}\PY{p}{,} \PY{l+m+mi}{100}\PY{p}{,} \PY{l+m+mi}{1000}\PY{p}{]}\PY{p}{)}
        \PY{n+nb}{print}\PY{p}{(}\PY{l+s+s2}{\PYZdq{}}\PY{l+s+s2}{Logaritmos}\PY{l+s+s2}{\PYZdq{}}\PY{p}{)}
        \PY{n+nb}{print}\PY{p}{(}\PY{n}{np}\PY{o}{.}\PY{n}{log10}\PY{p}{(}\PY{n}{b}\PY{p}{)}\PY{p}{)}                 \PY{c+c1}{\PYZsh{} Imprime el logaritmo en base 10 de cada elemento}
\end{Verbatim}


    \begin{center}\rule{0.5\linewidth}{\linethickness}\end{center}

 Funciones lógicas

\begin{itemize}
\tightlist
\item
  \texttt{logical\_and} (AND)
\item
  \texttt{logical\_or} (OR)
\item
  \texttt{logical\_not} (NOT)
\item
  \texttt{logical\_xor} (XOR)
\end{itemize}

    \begin{Verbatim}[commandchars=\\\{\}]
{\color{incolor}In [{\color{incolor} }]:} \PY{n}{m1} \PY{o}{=} \PY{n}{np}\PY{o}{.}\PY{n}{array}\PY{p}{(}\PY{p}{[}\PY{p}{[}\PY{k+kc}{False}\PY{p}{,} \PY{k+kc}{True}\PY{p}{]}\PY{p}{,}\PY{p}{[}\PY{k+kc}{False}\PY{p}{,} \PY{k+kc}{True}\PY{p}{]}\PY{p}{]}\PY{p}{,} \PY{n}{dtype}\PY{o}{=}\PY{n}{np}\PY{o}{.}\PY{n}{bool}\PY{p}{)}
        \PY{n}{m2} \PY{o}{=} \PY{n}{np}\PY{o}{.}\PY{n}{array}\PY{p}{(}\PY{p}{[}\PY{p}{[}\PY{k+kc}{True}\PY{p}{,} \PY{k+kc}{True}\PY{p}{]}\PY{p}{,}\PY{p}{[}\PY{k+kc}{False}\PY{p}{,} \PY{k+kc}{True}\PY{p}{]}\PY{p}{]}\PY{p}{,} \PY{n}{dtype}\PY{o}{=}\PY{n}{np}\PY{o}{.}\PY{n}{bool}\PY{p}{)}
        \PY{n+nb}{print}\PY{p}{(}\PY{n}{m1}\PY{p}{)}
        \PY{n+nb}{print}\PY{p}{(}\PY{n}{m2}\PY{p}{)}
        \PY{n+nb}{print}\PY{p}{(}\PY{p}{)}
        
        \PY{n+nb}{print}\PY{p}{(}\PY{n}{np}\PY{o}{.}\PY{n}{logical\PYZus{}and}\PY{p}{(}\PY{n}{m1}\PY{p}{,}\PY{n}{m2}\PY{p}{)}\PY{p}{)}                       \PY{c+c1}{\PYZsh{} Imprime un AND entre las dos matrices.}
        \PY{n+nb}{print}\PY{p}{(}\PY{n}{np}\PY{o}{.}\PY{n}{logical\PYZus{}xor}\PY{p}{(}\PY{n}{m1}\PY{p}{,}\PY{n}{m2}\PY{p}{)}\PY{p}{)}                       \PY{c+c1}{\PYZsh{} Imprime un XOR entre las dos matrices.}
\end{Verbatim}


    \begin{center}\rule{0.5\linewidth}{\linethickness}\end{center}

 Funciones y operaciones para comparación

\emph{Numpy} proporciona una serie de funciones para la comparación de
arrays, algunas de las cuales también se implementan como operador.

\begin{itemize}
\tightlist
\item
  \texttt{greater()} x1 \textgreater{} x2
\item
  \texttt{greater\_equal()} x1 \textgreater{}= x2
\item
  \texttt{less()} x1 \textless{} x2
\item
  \texttt{less\_equal()} x1 =\textless{} x2
\item
  \texttt{not\_equal()} x1 != x2
\item
  \texttt{equal()} x1 == x2
\item
  \texttt{maximum()} maximo(x1,x2)
\item
  \texttt{minimum()} minimo(x1,x2)
\end{itemize}

\emph{Nota}: No confundir \texttt{np.maximum} y \texttt{np.minimum} con
\texttt{np.max} y \texttt{np.min}. Mientras que los primeros realizan
operaciones elemento a elemento entre dos vectores, devolviendo un nuevo
vector, los segundos reciben un solo array y devuelven un único valor.

    \begin{Verbatim}[commandchars=\\\{\}]
{\color{incolor}In [{\color{incolor} }]:} \PY{n}{m1} \PY{o}{=} \PY{n}{np}\PY{o}{.}\PY{n}{array}\PY{p}{(}\PY{p}{[}\PY{p}{[}\PY{l+m+mi}{0}\PY{p}{,} \PY{l+m+mi}{1}\PY{p}{]}\PY{p}{,}\PY{p}{[}\PY{l+m+mi}{0}\PY{p}{,} \PY{l+m+mi}{3}\PY{p}{]}\PY{p}{]}\PY{p}{)}
        \PY{n}{m2} \PY{o}{=} \PY{n}{np}\PY{o}{.}\PY{n}{array}\PY{p}{(}\PY{p}{[}\PY{p}{[}\PY{l+m+mi}{1}\PY{p}{,} \PY{l+m+mi}{2}\PY{p}{]}\PY{p}{,}\PY{p}{[}\PY{l+m+mi}{0}\PY{p}{,} \PY{l+m+mi}{1}\PY{p}{]}\PY{p}{]}\PY{p}{)}
        \PY{n+nb}{print}\PY{p}{(}\PY{n}{m1}\PY{p}{)}
        \PY{n+nb}{print}\PY{p}{(}\PY{n}{m2}\PY{p}{)}
        \PY{n+nb}{print}\PY{p}{(}\PY{p}{)}
        
        \PY{n+nb}{print}\PY{p}{(}\PY{n}{m1}\PY{o}{\PYZgt{}}\PY{n}{m2}\PY{p}{)}
        \PY{n+nb}{print}\PY{p}{(}\PY{p}{)}
        
        \PY{n}{mm} \PY{o}{=} \PY{n}{np}\PY{o}{.}\PY{n}{maximum}\PY{p}{(}\PY{n}{m1}\PY{p}{,}\PY{n}{m2}\PY{p}{)}
        \PY{n+nb}{print}\PY{p}{(}\PY{n}{mm}\PY{p}{)}
\end{Verbatim}


    \begin{center}\rule{0.5\linewidth}{\linethickness}\end{center}

 Funciones específicas de punto flotante

\begin{itemize}
\tightlist
\item
  \texttt{around()}, \texttt{floor()}, \texttt{ceil()} (Redondeo)
\item
  \texttt{isfinite()} (Testea si los elementos son finitos, es decir, si
  no son infinitos, o no son \emph{NaN})
\item
  \texttt{isinf()} (Testea si los elementos son infinito positivo o
  negativo)
\item
  \texttt{isnan(x)} (Testea si los elementos son \emph{NaN})
\item
  \texttt{fabs(x)} (Devuelve el valor absoluto de los elementos)
\end{itemize}

    \begin{Verbatim}[commandchars=\\\{\}]
{\color{incolor}In [{\color{incolor} }]:} \PY{n}{m} \PY{o}{=} \PY{n}{np}\PY{o}{.}\PY{n}{array}\PY{p}{(}\PY{p}{[}\PY{p}{[}\PY{l+m+mf}{2.6}\PY{p}{,} \PY{l+m+mf}{3.2}\PY{p}{]}\PY{p}{,} \PY{p}{[}\PY{l+m+mf}{4.5}\PY{p}{,} \PY{l+m+mf}{5.6}\PY{p}{]}\PY{p}{]}\PY{p}{)}
        \PY{n+nb}{print}\PY{p}{(}\PY{n}{m}\PY{p}{)}
        \PY{n+nb}{print}\PY{p}{(}\PY{p}{)}
        
        \PY{n}{a} \PY{o}{=} \PY{n}{np}\PY{o}{.}\PY{n}{around}\PY{p}{(}\PY{n}{m}\PY{p}{)}              \PY{c+c1}{\PYZsh{} Redondeo al entero más cercano}
        \PY{n+nb}{print}\PY{p}{(}\PY{n}{a}\PY{p}{)}
        \PY{n+nb}{print}\PY{p}{(}\PY{p}{)}
        
        \PY{n}{f} \PY{o}{=} \PY{n}{np}\PY{o}{.}\PY{n}{floor}\PY{p}{(}\PY{n}{m}\PY{p}{)}               \PY{c+c1}{\PYZsh{} Redondeo al entero anterior}
        \PY{n+nb}{print}\PY{p}{(}\PY{n}{f}\PY{p}{)}
        \PY{n+nb}{print}\PY{p}{(}\PY{p}{)}
\end{Verbatim}


    \begin{center}\rule{0.5\linewidth}{\linethickness}\end{center}

 Operaciones "in place"

Es posible llevar a cabo las operaciones sobre vectores sin generar
copias adicionales. Esta funcionalidad puede resultar de interés cuando
el tamaño de los vectores es una limitación.

    \begin{Verbatim}[commandchars=\\\{\}]
{\color{incolor}In [{\color{incolor} }]:} \PY{n}{a} \PY{o}{=} \PY{n}{np}\PY{o}{.}\PY{n}{ones}\PY{p}{(}\PY{l+m+mi}{3}\PY{p}{)}\PY{o}{*}\PY{l+m+mi}{1}
        \PY{n+nb}{print}\PY{p}{(}\PY{n}{a}\PY{p}{)}
        \PY{n}{b} \PY{o}{=} \PY{n}{np}\PY{o}{.}\PY{n}{ones}\PY{p}{(}\PY{l+m+mi}{3}\PY{p}{)}\PY{o}{*}\PY{l+m+mi}{2}
        \PY{n+nb}{print}\PY{p}{(}\PY{n}{b}\PY{p}{)}
        \PY{n}{np}\PY{o}{.}\PY{n}{add}\PY{p}{(}\PY{n}{a}\PY{p}{,}\PY{n}{b}\PY{p}{,}\PY{n}{out}\PY{o}{=}\PY{n}{b}\PY{p}{)}
        \PY{n+nb}{print}\PY{p}{(}\PY{n}{b}\PY{p}{)}
\end{Verbatim}


     Esta forma de operar requiere un cuidado especial con las secuencias de
operaciones.

    \begin{center}\rule{0.5\linewidth}{\linethickness}\end{center}

 Ejercicio

Crear un vector \(x\) tamaño 100 con números separados de manera
equidistante en el intervalo \([-10,10]\). Generar otro vector con el
valor de la función sigmoide (\(\frac{1}{1+e^{-x}}\)) para cada uno de
ellos.

    \begin{Verbatim}[commandchars=\\\{\}]
{\color{incolor}In [{\color{incolor} }]:} \PY{o}{\PYZpc{}\PYZpc{}}\PY{k}{timeit}
        x = np.linspace(\PYZhy{}10,10,100)
        sig\PYZus{}x = 1/(1+np.exp(\PYZhy{}x))
        \PYZsh{} x = 
        \PYZsh{} sig\PYZus{}x = 
\end{Verbatim}


     Este tipo de operación se hace de manera muy frecuente para dibujar
gráficas con matplotlib (que veremos después). Por ejemplo:

    \begin{Verbatim}[commandchars=\\\{\}]
{\color{incolor}In [{\color{incolor} }]:} \PY{k+kn}{import} \PY{n+nn}{matplotlib} \PY{k}{as} \PY{n+nn}{mpl}                  \PY{c+c1}{\PYZsh{} Importa matplotlib y pyplot}
        \PY{k+kn}{import} \PY{n+nn}{matplotlib}\PY{n+nn}{.}\PY{n+nn}{pyplot} \PY{k}{as} \PY{n+nn}{plt}
        \PY{o}{\PYZpc{}}\PY{k}{matplotlib} inline                        
        
        \PY{n}{plt}\PY{o}{.}\PY{n}{plot}\PY{p}{(}\PY{n}{x}\PY{p}{,} \PY{n}{sig\PYZus{}x}\PY{p}{)}                       \PY{c+c1}{\PYZsh{} Dibuja una gráfica simple.}
\end{Verbatim}


    Generar un vector con 1000 números en el intervalo \([0,360]\) y
transformarlo a radianes. Obtener un array con el seno de cada uno de
ellos, y otro con el coseno.

    \begin{Verbatim}[commandchars=\\\{\}]
{\color{incolor}In [{\color{incolor} }]:} \PY{n}{x} \PY{o}{=} \PY{n}{np}\PY{o}{.}\PY{n}{linspace}\PY{p}{(}\PY{l+m+mi}{0}\PY{p}{,}\PY{l+m+mi}{360}\PY{p}{,}\PY{l+m+mi}{100}\PY{p}{)}
        \PY{n}{xr} \PY{o}{=} \PY{n}{x}\PY{o}{*}\PY{n}{np}\PY{o}{.}\PY{n}{pi}\PY{o}{/}\PY{l+m+mi}{180} 
        \PY{n}{sin\PYZus{}x} \PY{o}{=} \PY{n}{np}\PY{o}{.}\PY{n}{sin}\PY{p}{(}\PY{n}{xr}\PY{p}{)}
        \PY{n}{cos\PYZus{}x} \PY{o}{=} \PY{n}{np}\PY{o}{.}\PY{n}{cos}\PY{p}{(}\PY{n}{xr}\PY{p}{)}
        \PY{c+c1}{\PYZsh{} x = }
        \PY{c+c1}{\PYZsh{} xr =  }
        \PY{c+c1}{\PYZsh{} sin\PYZus{}x = }
        \PY{c+c1}{\PYZsh{} cos\PYZus{}x = }
\end{Verbatim}


    Lo dibujamos

    \begin{Verbatim}[commandchars=\\\{\}]
{\color{incolor}In [{\color{incolor} }]:} \PY{n}{plt}\PY{o}{.}\PY{n}{plot}\PY{p}{(}\PY{n}{x}\PY{p}{,} \PY{n}{sin\PYZus{}x}\PY{p}{,} \PY{n}{x}\PY{p}{,} \PY{n}{cos\PYZus{}x}\PY{p}{)}                 
\end{Verbatim}


     

    \begin{center}\rule{0.5\linewidth}{\linethickness}\end{center}

 8. Operaciones sobre matrices

\emph{NumPy} implementa funciones optimizadas (vectorizadas) para el
cálculo con arrays (no a nivel de elemento). Las de uso más común son el
producto, la inversa y la transposición.

     Producto vectorial

La función \texttt{np.dot()}implementa el producto vectorial entre dos
vectores o dos matrices. En el caso de dos vectores, el producto
vectorial se calcula como:

\[
[u_1, ... , u_n] \cdot \left[\begin{array}{c}
v_1 \cr
\ldots \cr
v_n
\end{array} \right] = u_1\cdot v_1 + \cdots + u_n \cdot v_n
\]

Por ejemplo:

\[
\left[\begin{array}{c c c} 1 & 2 & 3 \end{array} \right] \cdot \left[\begin{array}{c}
10 \cr
20 \cr
30
\end{array} \right] = 1\cdot 10 + 2\cdot 20 + 3 \cdot 30 = 140
\]

    \begin{Verbatim}[commandchars=\\\{\}]
{\color{incolor}In [{\color{incolor} }]:} \PY{n}{u} \PY{o}{=} \PY{n}{np}\PY{o}{.}\PY{n}{array}\PY{p}{(}\PY{p}{[}\PY{l+m+mi}{1}\PY{p}{,}\PY{l+m+mi}{2}\PY{p}{,}\PY{l+m+mi}{3}\PY{p}{]}\PY{p}{)}
        \PY{n}{v} \PY{o}{=} \PY{n}{np}\PY{o}{.}\PY{n}{array}\PY{p}{(}\PY{p}{[}\PY{l+m+mi}{10}\PY{p}{,}\PY{l+m+mi}{20}\PY{p}{,}\PY{l+m+mi}{30}\PY{p}{]}\PY{p}{)}
        
        \PY{n+nb}{print}\PY{p}{(}\PY{n}{np}\PY{o}{.}\PY{n}{dot}\PY{p}{(}\PY{n}{u}\PY{p}{,}\PY{n}{v}\PY{p}{)}\PY{p}{)}              \PY{c+c1}{\PYZsh{} Se puede llamar la función de las dos maneras. }
        \PY{n+nb}{print}\PY{p}{(}\PY{n}{u}\PY{o}{.}\PY{n}{dot}\PY{p}{(}\PY{n}{v}\PY{p}{)}\PY{p}{)}
\end{Verbatim}


    El producto vectorial de una matriz de tamaño (\(m \times n\)) y otra de
tamaño (\(n \times o\)), es una nueva matriz, de tamaño
(\(m \times o\)), en la que el valor de la posición (\(i,j\)) es
obtenido como el producto vectorial de la fila \(i\) de la primera
matriz, y la columna \(j\) de la segunda matriz.

\[
 \left[\begin{array}{c c c}
u_{11} & \cdots & u_{1n} \cr
\cdots & \cdots & \cdots\cr
u_{m1} &\cdots & u_{mn} \cr
\end{array} \right] \cdot  \left[\begin{array}{c c c}
v_{11} & \cdots & v_{1o} \cr
\cdots &\cdots & \cdots \cr
v_{n1} & \cdots & v_{no} \cr
\end{array} \right] =
\left[\begin{array}{c c c}
u_{11}\cdot v_{11} + \cdots + u_{1n}\cdot v_{n1}& \cdots & u_{11}\cdot v_{10} + \cdots + u_{1n}\cdot v_{n0}\cr
\cdots &\cdots & \cdots \cr
u_{m1}\cdot v_{11} + \cdots + u_{mn}\cdot v_{n1}& \cdots & u_{m1}\cdot v_{10} + \cdots + u_{mn}\cdot v_{n0} \cr
\end{array} \right]
\]

Por ejemplo: \[
\left[\begin{array}{c c}
1 & 2 \cr
3 & 4  \cr
5 & 6  \cr
\end{array} \right] \cdot
\left[\begin{array}{c c c c}
1&  10 & 100& 200\cr
2 & 4 &  6 & 8 \cr
\end{array} \right] = 
\left[\begin{array}{c c c c}
5 & 18 & 112& 216 \cr
11 & 46 & 324 & 632 \cr
17 & 74 & 536 & 1048 \cr
\end{array} \right]
\]

 Si el número de columnas de la primera matriz es distinto del número de
filas de la segunda, las matrices no se pueden multiplicar.

    \begin{Verbatim}[commandchars=\\\{\}]
{\color{incolor}In [{\color{incolor} }]:} \PY{n}{u} \PY{o}{=} \PY{n}{np}\PY{o}{.}\PY{n}{array}\PY{p}{(}\PY{p}{[}\PY{p}{[}\PY{l+m+mi}{1}\PY{p}{,}\PY{l+m+mi}{2}\PY{p}{]}\PY{p}{,}\PY{p}{[}\PY{l+m+mi}{3}\PY{p}{,}\PY{l+m+mi}{4}\PY{p}{]}\PY{p}{,}\PY{p}{[}\PY{l+m+mi}{5}\PY{p}{,}\PY{l+m+mi}{6}\PY{p}{]}\PY{p}{]}\PY{p}{)}
        \PY{n}{v} \PY{o}{=} \PY{n}{np}\PY{o}{.}\PY{n}{array}\PY{p}{(}\PY{p}{[}\PY{p}{[}\PY{l+m+mi}{1}\PY{p}{,}\PY{l+m+mi}{10}\PY{p}{,}\PY{l+m+mi}{100}\PY{p}{,}\PY{l+m+mi}{200}\PY{p}{]}\PY{p}{,}\PY{p}{[}\PY{l+m+mi}{2}\PY{p}{,}\PY{l+m+mi}{4}\PY{p}{,}\PY{l+m+mi}{6}\PY{p}{,}\PY{l+m+mi}{8}\PY{p}{]}\PY{p}{]}\PY{p}{)}
        
        \PY{n+nb}{print}\PY{p}{(}\PY{n}{np}\PY{o}{.}\PY{n}{dot}\PY{p}{(}\PY{n}{u}\PY{p}{,}\PY{n}{v}\PY{p}{)}\PY{p}{)}             \PY{c+c1}{\PYZsh{} Se puede llamar a la función de los dos modos. }
        \PY{n+nb}{print}\PY{p}{(}\PY{p}{)}
        \PY{n+nb}{print}\PY{p}{(}\PY{n}{u}\PY{o}{.}\PY{n}{dot}\PY{p}{(}\PY{n}{v}\PY{p}{)}\PY{p}{)}
\end{Verbatim}


     A partir de la versión 3.5 de Python, es posible utilizar el símbolo
'@' para llevar a cabo la multiplicación de matrices.

    \begin{Verbatim}[commandchars=\\\{\}]
{\color{incolor}In [{\color{incolor} }]:} \PY{n+nb}{print}\PY{p}{(}\PY{n}{u} \PY{o}{@} \PY{n}{v}\PY{p}{)}
\end{Verbatim}


    Cuando se utilizan matrices (objetos de tipo \texttt{matrix}), el
operador \texttt{*} se puede utilizar para hacer multiplicaciones
matriciales de modo similar a \texttt{dot()}

    \begin{Verbatim}[commandchars=\\\{\}]
{\color{incolor}In [{\color{incolor} }]:} \PY{n}{u} \PY{o}{=} \PY{n}{np}\PY{o}{.}\PY{n}{matrix}\PY{p}{(}\PY{l+s+s1}{\PYZsq{}}\PY{l+s+s1}{1,2;3,4;5,6}\PY{l+s+s1}{\PYZsq{}}\PY{p}{)}
        \PY{n}{v} \PY{o}{=} \PY{n}{np}\PY{o}{.}\PY{n}{matrix}\PY{p}{(}\PY{l+s+s1}{\PYZsq{}}\PY{l+s+s1}{1,10,100,200;2,4,6,8}\PY{l+s+s1}{\PYZsq{}}\PY{p}{)}
        
        \PY{n+nb}{print}\PY{p}{(}\PY{n}{np}\PY{o}{.}\PY{n}{dot}\PY{p}{(}\PY{n}{u}\PY{p}{,}\PY{n}{v}\PY{p}{)}\PY{p}{)}
        \PY{n+nb}{print}\PY{p}{(}\PY{p}{)}
        
        \PY{n+nb}{print}\PY{p}{(}\PY{n}{u}\PY{o}{*}\PY{n}{v}\PY{p}{)}
\end{Verbatim}


    \begin{center}\rule{0.5\linewidth}{\linethickness}\end{center}

 Inversa

La función \texttt{numpy.linalg.inv()} devuelve la inversa de una
matriz.

    \begin{Verbatim}[commandchars=\\\{\}]
{\color{incolor}In [{\color{incolor} }]:} \PY{n}{m} \PY{o}{=} \PY{n}{np}\PY{o}{.}\PY{n}{array}\PY{p}{(}\PY{p}{[}\PY{p}{[}\PY{l+m+mi}{1}\PY{p}{,}\PY{l+m+mi}{1}\PY{p}{,}\PY{l+m+mi}{1}\PY{p}{]}\PY{p}{,}\PY{p}{[}\PY{l+m+mi}{0}\PY{p}{,}\PY{l+m+mi}{2}\PY{p}{,}\PY{l+m+mi}{5}\PY{p}{]}\PY{p}{,}\PY{p}{[}\PY{l+m+mi}{2}\PY{p}{,}\PY{l+m+mi}{5}\PY{p}{,}\PY{o}{\PYZhy{}}\PY{l+m+mi}{1}\PY{p}{]}\PY{p}{]}\PY{p}{)} 
        
        \PY{n}{m\PYZus{}inv} \PY{o}{=} \PY{n}{np}\PY{o}{.}\PY{n}{linalg}\PY{o}{.}\PY{n}{inv}\PY{p}{(}\PY{n}{m}\PY{p}{)}
        \PY{n+nb}{print}\PY{p}{(}\PY{n}{m\PYZus{}inv}\PY{p}{)}
        \PY{n+nb}{print}\PY{p}{(}\PY{p}{)}
        
        \PY{c+c1}{\PYZsh{} El producto de una matriz por su inversa devuelve la matriz identidad}
        \PY{n+nb}{print}\PY{p}{(}\PY{n}{np}\PY{o}{.}\PY{n}{dot}\PY{p}{(}\PY{n}{m}\PY{p}{,}\PY{n}{m\PYZus{}inv}\PY{p}{)}\PY{p}{)}               \PY{c+c1}{\PYZsh{} Por precisión numérica, aparecen algunos decimales ínfimos. }
\end{Verbatim}


    Aunque no recomendamos abusar de la importación de paquetes y módulos
usando \texttt{from\ module\ import\ function}, si no se pierde
información de dónde viene cada módulo o función, se puede utilizar para
simplificar el aspecto del código.

    \begin{Verbatim}[commandchars=\\\{\}]
{\color{incolor}In [{\color{incolor} }]:} \PY{k+kn}{from} \PY{n+nn}{numpy}\PY{n+nn}{.}\PY{n+nn}{linalg} \PY{k}{import} \PY{n}{inv} \PY{k}{as} \PY{n}{inverse}
        
        \PY{n}{m\PYZus{}inv} \PY{o}{=} \PY{n}{inverse}\PY{p}{(}\PY{n}{m}\PY{p}{)}
        \PY{n+nb}{print}\PY{p}{(}\PY{n}{m\PYZus{}inv}\PY{p}{)}
\end{Verbatim}


    \begin{center}\rule{0.5\linewidth}{\linethickness}\end{center}

 Transposición de matrices

Aunque existen algunos otros métodos, \texttt{transpose} y
\texttt{ndarray.T} son los más importantes. Son equivalentes, y
\textbf{\emph{devuelven una vista}}. Por lo tanto, los elementos que se
cambien en la vista, se cambiarán en la matriz original.

    \begin{Verbatim}[commandchars=\\\{\}]
{\color{incolor}In [{\color{incolor} }]:} \PY{n}{a} \PY{o}{=} \PY{n}{np}\PY{o}{.}\PY{n}{array}\PY{p}{(}\PY{p}{[}\PY{p}{[}\PY{l+m+mi}{0}\PY{p}{,}\PY{l+m+mi}{1}\PY{p}{,}\PY{l+m+mi}{2}\PY{p}{,}\PY{l+m+mi}{3}\PY{p}{]}\PY{p}{,}\PY{p}{[}\PY{l+m+mi}{4}\PY{p}{,}\PY{l+m+mi}{5}\PY{p}{,}\PY{l+m+mi}{6}\PY{p}{,}\PY{l+m+mi}{7}\PY{p}{]}\PY{p}{,}\PY{p}{[}\PY{l+m+mi}{8}\PY{p}{,}\PY{l+m+mi}{9}\PY{p}{,}\PY{l+m+mi}{10}\PY{p}{,}\PY{l+m+mi}{11}\PY{p}{]}\PY{p}{]}\PY{p}{)}
        \PY{n+nb}{print}\PY{p}{(}\PY{n}{a}\PY{p}{)}
        \PY{n+nb}{print}\PY{p}{(}\PY{p}{)}
        
        \PY{n}{a\PYZus{}t} \PY{o}{=} \PY{n}{a}\PY{o}{.}\PY{n}{transpose}\PY{p}{(}\PY{p}{)}
        \PY{n+nb}{print}\PY{p}{(}\PY{n}{a\PYZus{}t}\PY{p}{)}
        \PY{n+nb}{print}\PY{p}{(}\PY{p}{)}
        
        \PY{n}{a\PYZus{}t}\PY{p}{[}\PY{l+m+mi}{0}\PY{p}{,}\PY{l+m+mi}{0}\PY{p}{]}\PY{o}{=}\PY{l+m+mi}{10}
        \PY{n+nb}{print}\PY{p}{(}\PY{n}{a}\PY{p}{)}
\end{Verbatim}


    \begin{center}\rule{0.5\linewidth}{\linethickness}\end{center}

 Extracción/creación de la diagonal de una matriz

La función \texttt{diag} permite extraer la diagonal de una matriz.
Admite un parámetro que permite especificar la diagonal.

    \begin{Verbatim}[commandchars=\\\{\}]
{\color{incolor}In [{\color{incolor} }]:} \PY{n}{a} \PY{o}{=} \PY{n}{np}\PY{o}{.}\PY{n}{array}\PY{p}{(}\PY{p}{[}\PY{p}{[}\PY{l+m+mi}{0}\PY{p}{,}\PY{l+m+mi}{1}\PY{p}{,}\PY{l+m+mi}{2}\PY{p}{,}\PY{l+m+mi}{3}\PY{p}{]}\PY{p}{,}\PY{p}{[}\PY{l+m+mi}{4}\PY{p}{,}\PY{l+m+mi}{5}\PY{p}{,}\PY{l+m+mi}{6}\PY{p}{,}\PY{l+m+mi}{7}\PY{p}{]}\PY{p}{,}\PY{p}{[}\PY{l+m+mi}{8}\PY{p}{,}\PY{l+m+mi}{9}\PY{p}{,}\PY{l+m+mi}{10}\PY{p}{,}\PY{l+m+mi}{11}\PY{p}{]}\PY{p}{,}\PY{p}{[}\PY{l+m+mi}{12}\PY{p}{,}\PY{l+m+mi}{13}\PY{p}{,}\PY{l+m+mi}{14}\PY{p}{,}\PY{l+m+mi}{15}\PY{p}{]}\PY{p}{]}\PY{p}{)}
        \PY{n+nb}{print}\PY{p}{(}\PY{n}{a}\PY{p}{)}
        \PY{n+nb}{print}\PY{p}{(}\PY{p}{)}
        \PY{n+nb}{print}\PY{p}{(}\PY{n}{np}\PY{o}{.}\PY{n}{diag}\PY{p}{(}\PY{n}{a}\PY{p}{)}\PY{p}{)}
        \PY{n+nb}{print}\PY{p}{(}\PY{n}{np}\PY{o}{.}\PY{n}{diag}\PY{p}{(}\PY{n}{a}\PY{p}{,}\PY{l+m+mi}{1}\PY{p}{)}\PY{p}{)}
        \PY{n+nb}{print}\PY{p}{(}\PY{n}{np}\PY{o}{.}\PY{n}{diag}\PY{p}{(}\PY{n}{a}\PY{p}{,}\PY{o}{\PYZhy{}}\PY{l+m+mi}{1}\PY{p}{)}\PY{p}{)}
\end{Verbatim}


    Si a la función anterior se le pasa un vector, entonces crea una matriz
diagonal.

    \begin{Verbatim}[commandchars=\\\{\}]
{\color{incolor}In [{\color{incolor} }]:} \PY{n+nb}{print}\PY{p}{(}\PY{n}{np}\PY{o}{.}\PY{n}{diag}\PY{p}{(}\PY{n}{np}\PY{o}{.}\PY{n}{array}\PY{p}{(}\PY{p}{[}\PY{l+m+mi}{1}\PY{p}{,}\PY{l+m+mi}{2}\PY{p}{,}\PY{l+m+mi}{3}\PY{p}{,}\PY{l+m+mi}{4}\PY{p}{]}\PY{p}{)}\PY{p}{)}\PY{p}{)}
\end{Verbatim}


    \begin{center}\rule{0.5\linewidth}{\linethickness}\end{center}

 Ejercicio

Generar una matriz aleatoria de tamaño \(5\times5\), su inversa, y
multiplicarlas. Redondear el resultado.

    \begin{Verbatim}[commandchars=\\\{\}]
{\color{incolor}In [{\color{incolor} }]:} \PY{c+c1}{\PYZsh{} Completar}
\end{Verbatim}


     

    \begin{center}\rule{0.5\linewidth}{\linethickness}\end{center}

 9. Otras funciones de interés

 Estas funciones no son universales, pero son de uso común.

    \begin{center}\rule{0.5\linewidth}{\linethickness}\end{center}

 Funciones generales

\begin{itemize}
\tightlist
\item
  \texttt{max()}, \texttt{min()}: Elemento máximo o mínimo.
\item
  \texttt{argmin()}, \texttt{argmax()}: Índice del elemento máximo o
  mínimo.
\item
  \texttt{sum()}: Suma de los elementos del array.
\item
  \texttt{cumsum()}: Suma acumulada de los elementos del array.
\item
  \texttt{prod()}: Producto de los elementos del array.
\item
  \texttt{cumprod()}: Producto acumulado de los elementos del array.
\item
  etc.
\end{itemize}

Todas ellas se aplican sobre el array, o sobre un eje que es
especificado como parámetro.

    \begin{Verbatim}[commandchars=\\\{\}]
{\color{incolor}In [{\color{incolor} }]:} \PY{n}{m} \PY{o}{=} \PY{n}{np}\PY{o}{.}\PY{n}{array}\PY{p}{(}\PY{p}{[}\PY{p}{[}\PY{l+m+mi}{3}\PY{p}{,}\PY{l+m+mi}{7}\PY{p}{,}\PY{l+m+mi}{5}\PY{p}{]}\PY{p}{,}\PY{p}{[}\PY{l+m+mi}{8}\PY{p}{,}\PY{l+m+mi}{4}\PY{p}{,}\PY{l+m+mi}{1}\PY{p}{]}\PY{p}{,}\PY{p}{[}\PY{l+m+mi}{2}\PY{p}{,}\PY{l+m+mi}{4}\PY{p}{,}\PY{l+m+mi}{9}\PY{p}{]}\PY{p}{,}\PY{p}{[}\PY{l+m+mi}{3}\PY{p}{,}\PY{l+m+mi}{1}\PY{p}{,}\PY{l+m+mi}{6}\PY{p}{]}\PY{p}{]}\PY{p}{)} 
        \PY{n+nb}{print}\PY{p}{(}\PY{n}{m}\PY{p}{)}
        \PY{n+nb}{print}\PY{p}{(}\PY{l+s+s2}{\PYZdq{}}\PY{l+s+s2}{Mínimo:}\PY{l+s+s2}{\PYZdq{}}\PY{p}{,}\PY{n}{np}\PY{o}{.}\PY{n}{amin}\PY{p}{(}\PY{n}{m}\PY{p}{)}\PY{p}{)}
        \PY{n+nb}{print}\PY{p}{(}\PY{l+s+s2}{\PYZdq{}}\PY{l+s+s2}{Mínimo de cada columna:}\PY{l+s+s2}{\PYZdq{}}\PY{p}{,} \PY{n}{np}\PY{o}{.}\PY{n}{amin}\PY{p}{(}\PY{n}{m}\PY{p}{,}\PY{l+m+mi}{0}\PY{p}{)}\PY{p}{)}
        \PY{n+nb}{print}\PY{p}{(}\PY{l+s+s2}{\PYZdq{}}\PY{l+s+s2}{Mínimo de cada fila:}\PY{l+s+s2}{\PYZdq{}}\PY{p}{,} \PY{n}{np}\PY{o}{.}\PY{n}{amin}\PY{p}{(}\PY{n}{m}\PY{p}{,}\PY{l+m+mi}{1}\PY{p}{)}\PY{p}{)}
        
        \PY{n+nb}{print}\PY{p}{(}\PY{l+s+s2}{\PYZdq{}}\PY{l+s+s2}{Rango de valores en cada fila:}\PY{l+s+s2}{\PYZdq{}}\PY{p}{,}\PY{n}{np}\PY{o}{.}\PY{n}{ptp}\PY{p}{(}\PY{n}{m}\PY{p}{,}\PY{l+m+mi}{1}\PY{p}{)}\PY{p}{)}
        \PY{n+nb}{print}\PY{p}{(}\PY{l+s+s2}{\PYZdq{}}\PY{l+s+s2}{Suma de los valores de cada fila:}\PY{l+s+s2}{\PYZdq{}}\PY{p}{,} \PY{n}{np}\PY{o}{.}\PY{n}{sum}\PY{p}{(}\PY{n}{m}\PY{p}{,}\PY{l+m+mi}{1}\PY{p}{)}\PY{p}{)}
        \PY{n+nb}{print}\PY{p}{(}\PY{l+s+s2}{\PYZdq{}}\PY{l+s+s2}{Suma de los valores de cada columna:}\PY{l+s+s2}{\PYZdq{}}\PY{p}{,} \PY{n}{np}\PY{o}{.}\PY{n}{sum}\PY{p}{(}\PY{n}{m}\PY{p}{,}\PY{l+m+mi}{0}\PY{p}{)}\PY{p}{)}
\end{Verbatim}


    \begin{center}\rule{0.5\linewidth}{\linethickness}\end{center}

 Funciones estadísticas básicas

NumPy implementa algunas funciones estadísticas basicas

\begin{itemize}
\tightlist
\item
  \texttt{amin()}, \texttt{amax()}: Elemento máximo o mínimo.
\item
  \texttt{ptp()}: El rango de valores.
\item
  \texttt{mean()}, \texttt{average()}, \texttt{median()}: Media, media
  ponderada, mediana.
\item
  \texttt{percentile()}: Percentil
\item
  \texttt{var()}: Varianza
\item
  \texttt{cov()}: Matriz de covarianzas
\item
  etc.
\end{itemize}

    \begin{Verbatim}[commandchars=\\\{\}]
{\color{incolor}In [{\color{incolor} }]:} \PY{n}{m} \PY{o}{=} \PY{n}{np}\PY{o}{.}\PY{n}{array}\PY{p}{(}\PY{p}{[}\PY{p}{[}\PY{l+m+mi}{3}\PY{p}{,}\PY{l+m+mi}{7}\PY{p}{,}\PY{l+m+mi}{5}\PY{p}{]}\PY{p}{,}\PY{p}{[}\PY{l+m+mi}{8}\PY{p}{,}\PY{l+m+mi}{4}\PY{p}{,}\PY{l+m+mi}{1}\PY{p}{]}\PY{p}{,}\PY{p}{[}\PY{l+m+mi}{2}\PY{p}{,}\PY{l+m+mi}{4}\PY{p}{,}\PY{l+m+mi}{9}\PY{p}{]}\PY{p}{,}\PY{p}{[}\PY{l+m+mi}{3}\PY{p}{,}\PY{l+m+mi}{1}\PY{p}{,}\PY{l+m+mi}{6}\PY{p}{]}\PY{p}{]}\PY{p}{)} 
        \PY{n+nb}{print}\PY{p}{(}\PY{n}{m}\PY{p}{,} \PY{l+s+s1}{\PYZsq{}}\PY{l+s+se}{\PYZbs{}n}\PY{l+s+s1}{\PYZsq{}}\PY{p}{)}
        \PY{n+nb}{print}\PY{p}{(}\PY{l+s+s2}{\PYZdq{}}\PY{l+s+s2}{Media de los valores de cada columna:}\PY{l+s+s2}{\PYZdq{}}\PY{p}{,} \PY{n}{np}\PY{o}{.}\PY{n}{mean}\PY{p}{(}\PY{n}{m}\PY{p}{,} \PY{n}{axis}\PY{o}{=}\PY{l+m+mi}{0}\PY{p}{)}\PY{p}{)}
        \PY{n+nb}{print}\PY{p}{(}\PY{l+s+s2}{\PYZdq{}}\PY{l+s+s2}{Desviación estándard de los valores de la matriz:}\PY{l+s+s2}{\PYZdq{}}\PY{p}{,} \PY{n}{np}\PY{o}{.}\PY{n}{sqrt}\PY{p}{(}\PY{n}{np}\PY{o}{.}\PY{n}{var}\PY{p}{(}\PY{n}{m}\PY{p}{)}\PY{p}{)}\PY{p}{)}
\end{Verbatim}


    \begin{center}\rule{0.5\linewidth}{\linethickness}\end{center}

 Funciones sobre conjuntos

NumPy proporciona varias funciones para el trabajo con conjuntos:

\begin{itemize}
\tightlist
\item
  \texttt{unique()} Devuelve los elementos únicos de un array.
\item
  \texttt{union1d()} Devuele la union de dos arrays de una dimensión (si
  no, se convierten)
\item
  \texttt{intersect1d()}Devuelve la intersección de dos arrays de una
  dimensión (si no, se convierten)
\item
  \texttt{in1d()} Permite determinar qué elementos del primer array
  están en el segundo.
\item
  etc.
\end{itemize}

    \begin{Verbatim}[commandchars=\\\{\}]
{\color{incolor}In [{\color{incolor} }]:} \PY{n}{a1} \PY{o}{=} \PY{n}{np}\PY{o}{.}\PY{n}{array}\PY{p}{(}\PY{p}{[}\PY{l+m+mi}{1}\PY{p}{,}\PY{l+m+mi}{2}\PY{p}{,}\PY{l+m+mi}{3}\PY{p}{,}\PY{l+m+mi}{4}\PY{p}{,}\PY{l+m+mi}{5}\PY{p}{,}\PY{l+m+mi}{6}\PY{p}{,}\PY{l+m+mi}{7}\PY{p}{,}\PY{l+m+mi}{8}\PY{p}{]}\PY{p}{)}
        \PY{n}{a2} \PY{o}{=} \PY{n}{np}\PY{o}{.}\PY{n}{array}\PY{p}{(}\PY{p}{[}\PY{l+m+mi}{1}\PY{p}{,}\PY{l+m+mi}{0}\PY{p}{,}\PY{l+m+mi}{1}\PY{p}{,}\PY{l+m+mi}{3}\PY{p}{,}\PY{l+m+mi}{0}\PY{p}{,}\PY{l+m+mi}{7}\PY{p}{]}\PY{p}{)}
        
        \PY{n+nb}{print}\PY{p}{(}\PY{n}{np}\PY{o}{.}\PY{n}{union1d}\PY{p}{(}\PY{n}{a1}\PY{p}{,}\PY{n}{a2}\PY{p}{)}\PY{p}{)}
        \PY{n+nb}{print}\PY{p}{(}\PY{n}{np}\PY{o}{.}\PY{n}{intersect1d}\PY{p}{(}\PY{n}{a1}\PY{p}{,}\PY{n}{a2}\PY{p}{)}\PY{p}{)}
        \PY{n+nb}{print}\PY{p}{(}\PY{n}{np}\PY{o}{.}\PY{n}{in1d}\PY{p}{(}\PY{n}{a2}\PY{p}{,} \PY{n}{a1}\PY{p}{)}\PY{p}{)}
\end{Verbatim}


    \begin{center}\rule{0.5\linewidth}{\linethickness}\end{center}

 Funciones sobre Strings

Las funciones para String llevan a cabo operaciones \emph{vectorizadas}
para arrays con tipo (\texttt{dtype}) (\texttt{numpy.string\_}) o
(\texttt{numpy.unicode\_}). Se basan en las funciones standard que
implementa \emph{Python}, que deben ser las utilizadas en la mayoría de
la situaciones. Éstas son algunas de las más importantes. -
\texttt{add()} Concatenación - \texttt{multiply()} Repetición -
\texttt{lower()} Convierte a minúscula - \texttt{upper()} Convierte los
elementos a mayúscula - \texttt{split()} Divide el String en partes. El
separador por defecto es el espacio, pero se le puede pasar un
separador. - \texttt{replace()} Permite reemplazar subcadenas en el
String

    \begin{Verbatim}[commandchars=\\\{\}]
{\color{incolor}In [{\color{incolor} }]:} \PY{n}{l} \PY{o}{=} \PY{n}{np}\PY{o}{.}\PY{n}{array}\PY{p}{(}\PY{p}{[}\PY{l+s+s2}{\PYZdq{}}\PY{l+s+s2}{Esta es la primera frase}\PY{l+s+s2}{\PYZdq{}}\PY{p}{,} \PY{l+s+s2}{\PYZdq{}}\PY{l+s+s2}{Aquí va la segunda}\PY{l+s+s2}{\PYZdq{}}\PY{p}{]}\PY{p}{)}
        \PY{n}{ls} \PY{o}{=} \PY{n}{np}\PY{o}{.}\PY{n}{chararray}\PY{o}{.}\PY{n}{split}\PY{p}{(}\PY{n}{l}\PY{p}{)}
        \PY{n+nb}{print}\PY{p}{(}\PY{n}{ls}\PY{p}{,} \PY{n}{ls}\PY{o}{.}\PY{n}{shape}\PY{p}{)}             \PY{c+c1}{\PYZsh{} La salida es otro vector con dos elementos. Cada uno es una lista.}
        \PY{n+nb}{print}\PY{p}{(}\PY{n}{ls}\PY{p}{[}\PY{l+m+mi}{1}\PY{p}{]}\PY{p}{)}                    \PY{c+c1}{\PYZsh{} Imprime la segunda lista}
        \PY{n+nb}{print}\PY{p}{(}\PY{n}{ls}\PY{p}{[}\PY{l+m+mi}{1}\PY{p}{]}\PY{p}{[}\PY{o}{\PYZhy{}}\PY{l+m+mi}{1}\PY{p}{]}\PY{p}{)}                \PY{c+c1}{\PYZsh{} Imprime la última palabra de la segunda lista.}
        \PY{n+nb}{print}\PY{p}{(}\PY{p}{)}
        
        \PY{n}{f} \PY{o}{=} \PY{p}{[}\PY{l+s+s2}{\PYZdq{}}\PY{l+s+s2}{Esto es una frase de ejemplo}\PY{l+s+s2}{\PYZdq{}}\PY{p}{]}
        \PY{n+nb}{print}\PY{p}{(}\PY{n}{np}\PY{o}{.}\PY{n}{chararray}\PY{o}{.}\PY{n}{replace}\PY{p}{(}\PY{n}{f}\PY{p}{,}\PY{l+s+s2}{\PYZdq{}}\PY{l+s+s2}{es}\PY{l+s+s2}{\PYZdq{}}\PY{p}{,}\PY{l+s+s2}{\PYZdq{}}\PY{l+s+s2}{puede ser}\PY{l+s+s2}{\PYZdq{}}\PY{p}{)}\PY{p}{)}      \PY{c+c1}{\PYZsh{} Reemplaza \PYZdq{}es\PYZdq{} por \PYZdq{}puede ser\PYZdq{}}
\end{Verbatim}


    \begin{center}\rule{0.5\linewidth}{\linethickness}\end{center}

 La librería SciPy

\emph{SciPy} es una librería que usa \emph{NumPy} como base, e
implementa una gran cantidad de funciones de utilidad relativas a
cálculo numérico, estadística, optimización etc. Existe una referencia
completa en la
\href{https://docs.scipy.org/doc/scipy/reference/}{documentación
oficial}.

Por ejemplo, el módulo
\href{https://docs.scipy.org/doc/scipy/reference/stats.html\#module-scipy.stats}{\texttt{scipy.stats}}
implementa una gran cantidad de distribuciones de probabilidad y
funciones estadísticas.

    \begin{Verbatim}[commandchars=\\\{\}]
{\color{incolor}In [{\color{incolor} }]:} \PY{k+kn}{from} \PY{n+nn}{scipy}\PY{n+nn}{.}\PY{n+nn}{stats} \PY{k}{import} \PY{n}{norm}
        
        \PY{n}{norm}\PY{o}{.}\PY{n}{cdf}\PY{p}{(}\PY{p}{[}\PY{o}{\PYZhy{}}\PY{l+m+mi}{3}\PY{p}{,} \PY{o}{\PYZhy{}}\PY{l+m+mi}{2}\PY{p}{,} \PY{o}{\PYZhy{}}\PY{l+m+mf}{1.}\PY{p}{,} \PY{l+m+mi}{0}\PY{p}{,} \PY{l+m+mi}{1}\PY{p}{,} \PY{l+m+mi}{2}\PY{p}{,} \PY{l+m+mi}{3}\PY{p}{]}\PY{p}{)}     \PY{c+c1}{\PYZsh{} Imprime la densidad acumulada para una lista de 7 valores en una distribución}
                                                \PY{c+c1}{\PYZsh{} normal (al no dar valor a ningún parámetro, se utiliza la estándar)}
\end{Verbatim}


    Si se utilizan arrays \emph{NumPy} en lugar de secuencias, muchas
operaciones implementadas en los objetos y funciones de SciPy se hacen
de forma vectorizada.

    \begin{Verbatim}[commandchars=\\\{\}]
{\color{incolor}In [{\color{incolor} }]:} \PY{n}{norm}\PY{o}{.}\PY{n}{cdf}\PY{p}{(}\PY{n}{np}\PY{o}{.}\PY{n}{array}\PY{p}{(}\PY{p}{[}\PY{o}{\PYZhy{}}\PY{l+m+mi}{3}\PY{p}{,} \PY{o}{\PYZhy{}}\PY{l+m+mi}{2}\PY{p}{,} \PY{o}{\PYZhy{}}\PY{l+m+mf}{1.}\PY{p}{,} \PY{l+m+mi}{0}\PY{p}{,} \PY{l+m+mi}{1}\PY{p}{,} \PY{l+m+mi}{2}\PY{p}{,} \PY{l+m+mi}{3}\PY{p}{]}\PY{p}{)}\PY{p}{)}
\end{Verbatim}


     En este tutorial, por no ser necesario salvo de manera puntual a lo
largo del curso, no se tratará SciPy

    \begin{center}\rule{0.5\linewidth}{\linethickness}\end{center}

 Ejercicio

Normaliza una matriz aleatoria de tamaño \(5\times4\) con números reales
del 0 al 10. (\(X_{norm} = \frac{X-X_{min}}{X_{max}-X_{min}}\)).

    \begin{Verbatim}[commandchars=\\\{\}]
{\color{incolor}In [{\color{incolor} }]:} \PY{c+c1}{\PYZsh{} Completar}
\end{Verbatim}


    Normalizar cada columna de forma independiente.

    \begin{Verbatim}[commandchars=\\\{\}]
{\color{incolor}In [{\color{incolor} }]:} \PY{c+c1}{\PYZsh{} Completar}
\end{Verbatim}


     

    \begin{center}\rule{0.5\linewidth}{\linethickness}\end{center}

 Búsqueda

La función \texttt{unique()} devuelve los elementos únicos en un array.
Es flexible, ya que se le puede indicar cómo devolver estos elementos.

    \begin{Verbatim}[commandchars=\\\{\}]
{\color{incolor}In [{\color{incolor} }]:} \PY{n}{v} \PY{o}{=} \PY{n}{np}\PY{o}{.}\PY{n}{array}\PY{p}{(}\PY{p}{[}\PY{l+m+mi}{5}\PY{p}{,}\PY{l+m+mi}{2}\PY{p}{,}\PY{l+m+mi}{6}\PY{p}{,}\PY{l+m+mi}{2}\PY{p}{,}\PY{l+m+mi}{7}\PY{p}{,}\PY{l+m+mi}{5}\PY{p}{,}\PY{l+m+mi}{6}\PY{p}{,}\PY{l+m+mi}{8}\PY{p}{,}\PY{l+m+mi}{2}\PY{p}{,}\PY{l+m+mi}{9}\PY{p}{]}\PY{p}{)} 
        \PY{n}{u} \PY{o}{=} \PY{n}{np}\PY{o}{.}\PY{n}{unique}\PY{p}{(}\PY{n}{v}\PY{p}{)} 
        \PY{n+nb}{print}\PY{p}{(}\PY{n}{u}\PY{p}{,}\PY{l+s+s1}{\PYZsq{}}\PY{l+s+se}{\PYZbs{}n}\PY{l+s+s1}{\PYZsq{}}\PY{p}{)}
        
        \PY{n}{u}\PY{p}{,}\PY{n}{indices} \PY{o}{=} \PY{n}{np}\PY{o}{.}\PY{n}{unique}\PY{p}{(}\PY{n}{v}\PY{p}{,} \PY{n}{return\PYZus{}index} \PY{o}{=} \PY{k+kc}{True}\PY{p}{)}   \PY{c+c1}{\PYZsh{} Se puede devolver también el vector, y los índices que ocupan los}
        \PY{n+nb}{print}\PY{p}{(}\PY{n}{indices}\PY{p}{,}\PY{l+s+s1}{\PYZsq{}}\PY{l+s+se}{\PYZbs{}n}\PY{l+s+s1}{\PYZsq{}}\PY{p}{)}                             \PY{c+c1}{\PYZsh{} elementos devueltos (la primera aparición)}
        
        \PY{n}{u}\PY{p}{,}\PY{n}{indices} \PY{o}{=} \PY{n}{np}\PY{o}{.}\PY{n}{unique}\PY{p}{(}\PY{n}{v}\PY{p}{,}\PY{n}{return\PYZus{}inverse} \PY{o}{=} \PY{k+kc}{True}\PY{p}{)}  \PY{c+c1}{\PYZsh{} Devuelve también un array con la correspondencia entre el array}
        \PY{n+nb}{print}\PY{p}{(}\PY{n}{indices}\PY{p}{,}\PY{l+s+s1}{\PYZsq{}}\PY{l+s+se}{\PYZbs{}n}\PY{l+s+s1}{\PYZsq{}}\PY{p}{)}                             \PY{c+c1}{\PYZsh{} original y el array devuelto.}
        
        \PY{n}{u}\PY{p}{,}\PY{n}{counts} \PY{o}{=} \PY{n}{np}\PY{o}{.}\PY{n}{unique}\PY{p}{(}\PY{n}{v}\PY{p}{,}\PY{n}{return\PYZus{}counts} \PY{o}{=} \PY{k+kc}{True}\PY{p}{)}    \PY{c+c1}{\PYZsh{} Devuelve un array con las veces que aparece cada elemento único}
        \PY{n+nb}{print}\PY{p}{(}\PY{n}{counts}\PY{p}{)}                                   \PY{c+c1}{\PYZsh{} El correspondiente a esa posición.}
\end{Verbatim}


    \emph{Numpy} proporciona algunas funciones para la localización de
elementos que cumplan determinados criterios. Por ejemplo, las funciones
\texttt{argmax()} y \texttt{argmin()} devuelven los índices de los
elementos mayor y menor, respectivamente, en el eje pasado como
parámetro.

    \begin{Verbatim}[commandchars=\\\{\}]
{\color{incolor}In [{\color{incolor} }]:} \PY{n}{m} \PY{o}{=} \PY{n}{np}\PY{o}{.}\PY{n}{array}\PY{p}{(}\PY{p}{[}\PY{p}{[}\PY{l+m+mi}{2} \PY{p}{,}\PY{l+m+mi}{6}\PY{p}{,} \PY{l+m+mi}{4}\PY{p}{]}\PY{p}{,} \PY{p}{[}\PY{l+m+mi}{4}\PY{p}{,}  \PY{l+m+mi}{3} \PY{p}{,} \PY{l+m+mi}{8}\PY{p}{]}\PY{p}{,} \PY{p}{[}\PY{l+m+mi}{1}\PY{p}{,} \PY{l+m+mi}{5}\PY{p}{,} \PY{l+m+mi}{2}\PY{p}{]}\PY{p}{,} \PY{p}{[}\PY{l+m+mi}{40}\PY{p}{,} \PY{l+m+mi}{2}\PY{p}{,} \PY{l+m+mi}{2}\PY{p}{]}\PY{p}{]}\PY{p}{)}
        \PY{n+nb}{print}\PY{p}{(}\PY{n}{m}\PY{p}{,}\PY{l+s+s1}{\PYZsq{}}\PY{l+s+se}{\PYZbs{}n}\PY{l+s+s1}{\PYZsq{}}\PY{p}{)}
        \PY{n+nb}{print}\PY{p}{(}\PY{n}{m}\PY{o}{.}\PY{n}{argmax}\PY{p}{(}\PY{p}{)}\PY{p}{,}\PY{l+s+s1}{\PYZsq{}}\PY{l+s+se}{\PYZbs{}n}\PY{l+s+s1}{\PYZsq{}}\PY{p}{)}                 \PY{c+c1}{\PYZsh{} Corresponde al índice en el array si el conjunto }
        \PY{n+nb}{print} \PY{p}{(}\PY{n}{m}\PY{o}{.}\PY{n}{flatten}\PY{p}{(}\PY{p}{)}\PY{p}{,}\PY{l+s+s1}{\PYZsq{}}\PY{l+s+se}{\PYZbs{}n}\PY{l+s+s1}{\PYZsq{}}\PY{p}{)}               \PY{c+c1}{\PYZsh{} de elementos se indexa como vector}
        
        \PY{n}{maxindices} \PY{o}{=} \PY{n}{np}\PY{o}{.}\PY{n}{argmax}\PY{p}{(}\PY{n}{m}\PY{p}{,} \PY{n}{axis} \PY{o}{=} \PY{l+m+mi}{0}\PY{p}{)}    \PY{c+c1}{\PYZsh{} Devuelve un array con los índices del elemento máximo}
        \PY{n+nb}{print}\PY{p}{(}\PY{n}{maxindices}\PY{p}{,} \PY{l+s+s1}{\PYZsq{}}\PY{l+s+se}{\PYZbs{}n}\PY{l+s+s1}{\PYZsq{}}\PY{p}{)}                \PY{c+c1}{\PYZsh{} en cada columna}
\end{Verbatim}


    La función \texttt{nonzero()} devuelve los índices de los elementos que
no son cero en el array. Devuelve un array con los índices en cada
dimensión.

    \begin{Verbatim}[commandchars=\\\{\}]
{\color{incolor}In [{\color{incolor} }]:} \PY{n}{m} \PY{o}{=} \PY{n}{np}\PY{o}{.}\PY{n}{array}\PY{p}{(}\PY{p}{[}\PY{p}{[}\PY{l+m+mi}{0} \PY{p}{,}\PY{l+m+mi}{6}\PY{p}{,} \PY{l+m+mi}{4}\PY{p}{]}\PY{p}{,} \PY{p}{[}\PY{l+m+mi}{0}\PY{p}{,}  \PY{l+m+mi}{3} \PY{p}{,} \PY{l+m+mi}{0}\PY{p}{]}\PY{p}{,} \PY{p}{[}\PY{l+m+mi}{0}\PY{p}{,} \PY{l+m+mi}{5}\PY{p}{,} \PY{l+m+mi}{2}\PY{p}{]}\PY{p}{,} \PY{p}{[}\PY{l+m+mi}{0}\PY{p}{,} \PY{l+m+mi}{2}\PY{p}{,} \PY{l+m+mi}{0}\PY{p}{]}\PY{p}{]}\PY{p}{)}
        \PY{n+nb}{print}\PY{p}{(}\PY{n}{m}\PY{p}{,}\PY{l+s+s1}{\PYZsq{}}\PY{l+s+se}{\PYZbs{}n}\PY{l+s+s1}{\PYZsq{}}\PY{p}{)}
        
        \PY{n}{f}\PY{p}{,}\PY{n}{c} \PY{o}{=} \PY{n}{m}\PY{o}{.}\PY{n}{nonzero}\PY{p}{(}\PY{p}{)}             \PY{c+c1}{\PYZsh{} 5 elementos no son cero.}
        \PY{n+nb}{print}\PY{p}{(}\PY{n}{f}\PY{p}{,} \PY{n}{c}\PY{p}{,}\PY{l+s+s1}{\PYZsq{}}\PY{l+s+se}{\PYZbs{}n}\PY{l+s+s1}{\PYZsq{}}\PY{p}{)}
        \PY{n+nb}{print}\PY{p}{(}\PY{l+s+s2}{\PYZdq{}}\PY{l+s+s2}{El primer elemento que no es cero es: [}\PY{l+s+si}{\PYZpc{}d}\PY{l+s+s2}{,}\PY{l+s+si}{\PYZpc{}d}\PY{l+s+s2}{]}\PY{l+s+s2}{\PYZdq{}} \PY{o}{\PYZpc{}} \PY{p}{(}\PY{n}{f}\PY{p}{[}\PY{l+m+mi}{0}\PY{p}{]}\PY{p}{,} \PY{n}{c}\PY{p}{[}\PY{l+m+mi}{0}\PY{p}{]}\PY{p}{)}\PY{p}{)}
        \PY{n+nb}{print}\PY{p}{(}\PY{l+s+s2}{\PYZdq{}}\PY{l+s+s2}{El último elemento que no es cero es: [}\PY{l+s+si}{\PYZpc{}d}\PY{l+s+s2}{,}\PY{l+s+si}{\PYZpc{}d}\PY{l+s+s2}{]}\PY{l+s+s2}{\PYZdq{}} \PY{o}{\PYZpc{}} \PY{p}{(}\PY{n}{f}\PY{p}{[}\PY{o}{\PYZhy{}}\PY{l+m+mi}{1}\PY{p}{]}\PY{p}{,} \PY{n}{c}\PY{p}{[}\PY{o}{\PYZhy{}}\PY{l+m+mi}{1}\PY{p}{]}\PY{p}{)}\PY{p}{)}
\end{Verbatim}


    La función \texttt{where()} devuelve los índices de los elementos en un
array que cumplen cierta condición.

    \begin{Verbatim}[commandchars=\\\{\}]
{\color{incolor}In [{\color{incolor} }]:} \PY{n}{m} \PY{o}{=} \PY{n}{np}\PY{o}{.}\PY{n}{array}\PY{p}{(}\PY{p}{[}\PY{p}{[}\PY{l+m+mi}{2} \PY{p}{,}\PY{l+m+mi}{6}\PY{p}{,} \PY{l+m+mi}{4}\PY{p}{]}\PY{p}{,} \PY{p}{[}\PY{l+m+mi}{4}\PY{p}{,}  \PY{l+m+mi}{3} \PY{p}{,} \PY{l+m+mi}{8}\PY{p}{]}\PY{p}{,} \PY{p}{[}\PY{l+m+mi}{1}\PY{p}{,} \PY{l+m+mi}{5}\PY{p}{,} \PY{l+m+mi}{2}\PY{p}{]}\PY{p}{,} \PY{p}{[}\PY{l+m+mi}{9}\PY{p}{,} \PY{l+m+mi}{2}\PY{p}{,} \PY{l+m+mi}{2}\PY{p}{]}\PY{p}{]}\PY{p}{)}
        \PY{n+nb}{print}\PY{p}{(}\PY{n}{m}\PY{p}{,}\PY{l+s+s1}{\PYZsq{}}\PY{l+s+se}{\PYZbs{}n}\PY{l+s+s1}{\PYZsq{}}\PY{p}{)}
        \PY{n+nb}{print}\PY{p}{(}\PY{l+s+s1}{\PYZsq{}}\PY{l+s+s1}{Posiciones de los elementos pares}\PY{l+s+s1}{\PYZsq{}}\PY{p}{)}
        \PY{n}{m\PYZus{}par} \PY{o}{=} \PY{n}{np}\PY{o}{.}\PY{n}{where}\PY{p}{(}\PY{n}{m}\PY{o}{\PYZpc{}}\PY{k}{2} == 0)   
        \PY{n+nb}{print}\PY{p}{(}\PY{n}{m\PYZus{}par}\PY{p}{)}
        \PY{n+nb}{print}\PY{p}{(}\PY{l+s+s2}{\PYZdq{}}\PY{l+s+se}{\PYZbs{}n}\PY{l+s+s2}{El primer elemento par es: [}\PY{l+s+si}{\PYZpc{}d}\PY{l+s+s2}{,}\PY{l+s+si}{\PYZpc{}d}\PY{l+s+s2}{]}\PY{l+s+s2}{\PYZdq{}} \PY{o}{\PYZpc{}} \PY{p}{(}\PY{n}{m\PYZus{}par}\PY{p}{[}\PY{l+m+mi}{0}\PY{p}{]}\PY{p}{[}\PY{l+m+mi}{0}\PY{p}{]}\PY{p}{,} \PY{n}{m\PYZus{}par}\PY{p}{[}\PY{l+m+mi}{1}\PY{p}{]}\PY{p}{[}\PY{l+m+mi}{0}\PY{p}{]}\PY{p}{)}\PY{p}{)}
\end{Verbatim}


    La función \texttt{any()} permite determinar si existe un elemento en el
array cuyo valor sea \texttt{True} (o distinto de 0).

    \begin{Verbatim}[commandchars=\\\{\}]
{\color{incolor}In [{\color{incolor} }]:} \PY{n}{np}\PY{o}{.}\PY{n}{any}\PY{p}{(}\PY{p}{[}\PY{p}{[}\PY{k+kc}{True}\PY{p}{,} \PY{k+kc}{False}\PY{p}{]}\PY{p}{,} \PY{p}{[}\PY{k+kc}{True}\PY{p}{,} \PY{k+kc}{True}\PY{p}{]}\PY{p}{,} \PY{p}{[}\PY{k+kc}{True}\PY{p}{,} \PY{k+kc}{False}\PY{p}{]}\PY{p}{]}\PY{p}{)}
\end{Verbatim}


    Esta función se utiliza de manera frecuente para comprobar si algún
elemento cumple una condición determinada. Por ejemplo, el siguiente
código comprueba si existe algún valor mayor que 900 en la matriz.

    \begin{Verbatim}[commandchars=\\\{\}]
{\color{incolor}In [{\color{incolor} }]:} \PY{n}{a} \PY{o}{=} \PY{n}{np}\PY{o}{.}\PY{n}{random}\PY{o}{.}\PY{n}{randint}\PY{p}{(}\PY{l+m+mi}{1000}\PY{p}{,} \PY{n}{size}\PY{o}{=}\PY{p}{(}\PY{l+m+mi}{10}\PY{p}{,} \PY{l+m+mi}{10}\PY{p}{)}\PY{p}{)}
        \PY{n+nb}{print}\PY{p}{(}\PY{n}{a}\PY{p}{)}
        \PY{n+nb}{print}\PY{p}{(}\PY{l+s+s2}{\PYZdq{}}\PY{l+s+se}{\PYZbs{}n}\PY{l+s+s2}{\PYZdq{}}\PY{p}{,}\PY{n}{np}\PY{o}{.}\PY{n}{any}\PY{p}{(}\PY{n}{a}\PY{o}{\PYZgt{}}\PY{l+m+mi}{900}\PY{p}{)}\PY{p}{)}
\end{Verbatim}


    La función \texttt{all()} determina si todos los elementos del array son
\texttt{True} distintos de cero. El siguiente código permite comprobar
si todos los elementos de la matriz son mayores que 50.

    \begin{Verbatim}[commandchars=\\\{\}]
{\color{incolor}In [{\color{incolor} }]:} \PY{n+nb}{print}\PY{p}{(}\PY{n}{np}\PY{o}{.}\PY{n}{all}\PY{p}{(}\PY{n}{a}\PY{o}{\PYZgt{}}\PY{l+m+mi}{50}\PY{p}{)}\PY{p}{)}
\end{Verbatim}


    \begin{center}\rule{0.5\linewidth}{\linethickness}\end{center}

Tests y comprobaciones

Numpy también implementa una serie de funciones que permiten hacer
comprobaciones sobre arrays.

\begin{itemize}
\tightlist
\item
  \texttt{isnan()} comprueba, para cada elemento, si es nan.
\item
  \texttt{isfinite()} comprueba si cada elemento es finito (distinto de
  infinito y distinto de np.nan)
\item
  \texttt{isinf()} comprueba si cada elemento es infinito.
\end{itemize}

    \begin{Verbatim}[commandchars=\\\{\}]
{\color{incolor}In [{\color{incolor} }]:} \PY{n}{a} \PY{o}{=} \PY{n}{np}\PY{o}{.}\PY{n}{ones}\PY{p}{(}\PY{p}{(}\PY{l+m+mi}{3}\PY{p}{,}\PY{l+m+mi}{3}\PY{p}{)}\PY{p}{)}
        \PY{n}{a}\PY{p}{[}\PY{l+m+mi}{1}\PY{p}{,}\PY{l+m+mi}{1}\PY{p}{]}\PY{o}{=}\PY{n}{np}\PY{o}{.}\PY{n}{nan}
        \PY{n+nb}{print}\PY{p}{(}\PY{n}{a}\PY{p}{)}
        \PY{n+nb}{print}\PY{p}{(}\PY{p}{)}
        
        \PY{n+nb}{print}\PY{p}{(}\PY{n}{np}\PY{o}{.}\PY{n}{isnan}\PY{p}{(}\PY{n}{a}\PY{p}{)}\PY{p}{)}
        \PY{n+nb}{print}\PY{p}{(}\PY{p}{)}
        \PY{n+nb}{print}\PY{p}{(}\PY{n}{np}\PY{o}{.}\PY{n}{isfinite}\PY{p}{(}\PY{n}{a}\PY{p}{)}\PY{p}{)}
\end{Verbatim}


    La llamada \texttt{a==np.nan} \textbf{no} devolvería el resultado
correcto porque las comparaciones que involucran elementos con valor
\texttt{np.nan} siempre devuelven \texttt{False}.

    \begin{center}\rule{0.5\linewidth}{\linethickness}\end{center}

Ordenación

La función \texttt{sort()} devuelve una copia con el contenido de un
array ordenado. Toma varios parémetros. Uno de ellos es la dimensión que
se ordena. Otro parámetro importante es el algoritmo de ordenación.
Implementa \emph{quicksort}, \emph{heapsort} y \emph{mergesort}. La
función \texttt{argsort()} es parecida, pero devuelve los índices de los
elementos en orden.

    \begin{Verbatim}[commandchars=\\\{\}]
{\color{incolor}In [{\color{incolor} }]:} \PY{n}{m} \PY{o}{=} \PY{n}{np}\PY{o}{.}\PY{n}{around}\PY{p}{(}\PY{n}{np}\PY{o}{.}\PY{n}{random}\PY{o}{.}\PY{n}{random}\PY{p}{(}\PY{p}{(}\PY{l+m+mi}{4}\PY{p}{,}\PY{l+m+mi}{5}\PY{p}{)}\PY{p}{)}\PY{o}{*}\PY{l+m+mi}{100}\PY{p}{)}     \PY{c+c1}{\PYZsh{} Crea una matriz de enteros del 0 al 100}
        \PY{n+nb}{print}\PY{p}{(}\PY{n}{m}\PY{p}{,}\PY{l+s+s1}{\PYZsq{}}\PY{l+s+se}{\PYZbs{}n}\PY{l+s+s1}{\PYZsq{}}\PY{p}{)}
        
        \PY{n}{ms} \PY{o}{=} \PY{n}{np}\PY{o}{.}\PY{n}{sort}\PY{p}{(}\PY{n}{m}\PY{p}{,} \PY{n}{axis}\PY{o}{=}\PY{l+m+mi}{0}\PY{p}{)}                        \PY{c+c1}{\PYZsh{} Por defecto, ordena cada columna.}
        \PY{n+nb}{print}\PY{p}{(}\PY{n}{ms}\PY{p}{,}\PY{l+s+s1}{\PYZsq{}}\PY{l+s+se}{\PYZbs{}n}\PY{l+s+s1}{\PYZsq{}}\PY{p}{)}
        
        \PY{n}{ms} \PY{o}{=} \PY{n}{np}\PY{o}{.}\PY{n}{sort}\PY{p}{(}\PY{n}{m}\PY{p}{,} \PY{n}{axis}\PY{o}{=}\PY{l+m+mi}{1}\PY{p}{)}                        \PY{c+c1}{\PYZsh{} Puede ordenar cada fila}
        \PY{n+nb}{print}\PY{p}{(}\PY{n}{ms}\PY{p}{,}\PY{l+s+s1}{\PYZsq{}}\PY{l+s+se}{\PYZbs{}n}\PY{l+s+s1}{\PYZsq{}}\PY{p}{)}
        
        \PY{n}{v} \PY{o}{=} \PY{n}{np}\PY{o}{.}\PY{n}{array}\PY{p}{(}\PY{p}{[}\PY{l+m+mi}{80}\PY{p}{,} \PY{l+m+mi}{40}\PY{p}{,} \PY{l+m+mi}{20}\PY{p}{,} \PY{l+m+mi}{30}\PY{p}{,} \PY{l+m+mi}{90}\PY{p}{]}\PY{p}{)}
        \PY{n}{msi} \PY{o}{=} \PY{n}{np}\PY{o}{.}\PY{n}{argsort}\PY{p}{(}\PY{n}{v}\PY{p}{)}
        \PY{n+nb}{print}\PY{p}{(}\PY{n}{v}\PY{p}{)}
        \PY{n+nb}{print}\PY{p}{(}\PY{n}{msi}\PY{p}{,}\PY{l+s+s1}{\PYZsq{}}\PY{l+s+se}{\PYZbs{}n}\PY{l+s+s1}{\PYZsq{}}\PY{p}{)}
\end{Verbatim}


    \begin{center}\rule{0.5\linewidth}{\linethickness}\end{center}

 


    % Add a bibliography block to the postdoc
    
    
    
    \end{document}
